\chapter{Discussion, Conclusion and Future Work}
\label{chapter:discussion}

\section{Discussion}
\label{sec:discussion}

The core of the disUnison model, the three strategies, are implemented in just a handful
lines of code and, thanks to being an extension of the Unison model, there is never a risk
of breaking the functionality of the generated code. This is not to say that every
diversification strategy allows for proper executable code to be generated, but thanks to
the nature of constraint solving the result will either be all possible solutions
(given enough execution time) or a proof that no solution, and thus no proper executable,
exists.

Regardless of what the selection of strategies may indicate the possibilities for
diversification are far broader than when approaching the problem in terms of register
allocation and instruction scheduling as separate procedures. It is important to keep in
mind that more unorthodox strategies that exploit the combined approach might be even
more performant. As mentioned in \ref{sec:unison-model} the Unison model consists of more
variables than the 4 explored in this experiment, all of which offers potential for
diversity. The strategies can also be combined, both in terms of combining the constraints
as well as combining the resulting function differently. E.g one could construct a program
version by combining function versions from different strategies and sampling rates.

Apart from cost, the set of active operations is the first variable to be assigned during
search (see Section \ref{sec:branch_strategy}). This partly explains the long execution
time of the code generator for the schedule strategy. When the issue cycles of a set of
active operations have been exhausted all branching decisions following the cost are
discarded. In contrast the registers strategy can keep the decisions relating to active
operations and which instruction implements which operation and the enumerate strategy can
keep all but the very last decision taken.

As an explanation for the effectiveness of the schedule strategy consider what was
mentioned in Section \ref{sec:schedule}. When applying the schedule strategy, not only can
the order of operations change between solutions, the number of operations can also change.
During search when applying the schedule strategy all combinations of issue cycles for a
valid set of active operations will be explored. When all possible orderings of that
particular set has been exhausted the search engine will attempt to find a new set of
active operations. This new set will then contain more or less operations than
the previous set. By introducing or removing instructions either all following instructions
are effectively shifted. Since code re-use attacks rely on the exact address of the
instructions to be executed, a simple shift can foil the attack. In addition to breaking
gadgets by simply swapping independent instructions, these shifts are likely the cause of
the effectiveness of the schedule strategy. However, inserting instructions will in most
cases mean a larger overhead.

There is a constant trade-off between diversity and execution overhead when generating
diverse populations of executables and the systematic approach is not excluded from this.
From the results we can deduce that there is a correlation that a more diverse population
in terms of gadgets incurs a wider distribution and a higher mean for the cost metric for
the implemented strategies. However, when comparing sampling rates 10 and 100 of the
enumerate and registers strategy the higher sampling rate generates a noticeably more
diverse population, but the mean cost in speed is virtually unchanged and the code size is
only increased by a handful of percentage points. Perhaps a more advanced strategy can
lessen this gap even more and still provide a diverse population of executables.

Unison (and disUnison) accepts the basic LLVM-solution as an optional parameter, and can
post constraints to only generate solutions that are \textit{better}. In other words, we
can generate executables with zero overhead \textit{with respect to LLVM's solution}.
Certain strategies would of course have an overhead, but with respect to the
\textit{optimal solution}. There is also the opportunity to limit the resulting population
in other ways, e.g only incur a 5\% overhead compared to the LLVM solution. By not
randomizing we have full control of the process and can limit the resulting code in
whatever way is appropriate. As this would limit the number of possible version this
optional parameter was not used during the experiment. It is however an exciting factor
to consider for future work.

\section{Threats to Validity}

\textcite{large-scale-automated} found 433.milc from the SPEC2006 benchmark suite to be
representative in terms of surviving gadgets. Unfortunately 433.milc was too large for the
experiment and the Unison test suite had to be used instead. Most of the functions in
433.milc were too large to find even one solution (one searching for about one hour).
This is an obvious problem for any practical purpose of the systematic approach. One
solution to this problem is to modify the search heuristics of the disUnison model even
further. Unfortunately it would be difficult, if not impossible, to find an optimal,
generally-applicable search strategy.

For a more proper comparison of the systematic approach tests would have to be repeated for
a more comprehensive data set and target the x86 or x86-64 architecture. As mentioned in
Section \ref{sec:arch} Unison in its current state does not support the x86 and x86-64
architecture and the tests were conducted for the Hexagon architecture. This is, of course,
not optimal but given that the strategies are not explicitly tied to the underlying
instructions and that they are potentially applied to all instructions the results are
still promising. If or when Unison or a similar tool implements support for the x86 and
x86-64 architecture, the experiment would have be repeated on to test whether or not the
strategies are equally performant.

\section{Conclusion}

Using a constraint solver to generate diverse binaries is an attractive approach given
the ease of implementation and the quality of the generated code. The resulting population
of diversified programs shows that the systematic approach has great potential at breaking
gadgets as well as providing great control of the incurred overhead. However, the
shortcomings of the experiment and the tool are a testament to the future work required.
The hypothesis remains unanswered.
