\chapter{Results}

\section{Surviving Gadgets}

\begin{figure}[htp]
	\centering
	\includegraphics[width=\textwidth,height=\textheight]{results/figures/gadgets}
	\caption{The ratio of occurence for each gadget broken down by strategy and sampling rate.
Each bar shows the occurence ratio for one particular gadget.}
	\label{fig:gadgets}
\end{figure}

Figure \ref{fig:gadgets} shows the occurence ratio of each gadget for the different
strategies and sampling rates. There were around 21000 gadgets in total for every program
across all versions. The x-axis shows a gadget id and there is no definitive correlation
between the gadgets across strategies and sampling rates. I.e gadget 0 for the enumerate
strategy at sampling rate 1 is not necessarily the same gadget 0 as the registers strategy
at sampling rate 10.

As seen in figure \ref{fig:gadgets}, neither the enumeration nor the registers strategy
were particularly effective at breaking gadgets for any sampling rate. There is a slight
improvement for higher sampling rates but it is not particularly impressive even at
sampling rate 1000. There are still many gadgets that survives between versions.

Curiously, there seems to be little difference between sampling rate 10 and 100 for both
enumerate and registers. Sampling rate of 100 appears to be even worse at breaking gadgets
than sampling rate 10. This is surprising and there is no definitive explanation for it.
I suspect this is because.... ?

As mentioned, \textcite{large-scale-automated} found that randomizing the instruction
schedule broke on average more than 95\% gadgets with respect to the original, undiversifed
binary. This would then mean that on average less than 5\% of gadgets survive the
diversification process and are present in all versions, compared to our results where
\textit{no} gadget is present is all versions. In fact, no gadget is present in even
50\% of versions. The strategy is even more effective at higher sampling rates, which is
in accordance with the expected behaviour described in section \ref{sec:sampling_rate}.


\section{Cost}

The estimated cost of each program is shown in Figure \ref{fig:cost}. The dotted red line
shows the cost of the LLVM solution when calculated in the same manner.

\begin{figure}[h]
	\centering
	\includegraphics[width=\textwidth,height=0.5\textheight]{results/figures/cost}
	\caption{The cost distributions for every strategy and sampling rate. The cost of the LLVM solution is included for reference.}
	\label{fig:cost}
\end{figure}

All strategies perform better for lower sampling rates. As described in section
\ref{sec:performance}, this is expected. Enumerate and registers perform equally well and
for sampling sizes of 1 and 10 they have a lower cost than the LLVM solution. The schedule
strategy seems to incur a slight overhead compared to the LLVM solution for all sampling
rates.

\begin{table}[h]
	\centering
	\begin{tabular}{rrrr}
\hline
   Sampling Rate &   Mean Est. Cost (cycles) &   Difference (cycles) &   Overhead (\textperthousand) \\
\hline
               1 &                  45299359 &                    22 &                      0.000486 \\
              10 &                  45299376 &                    39 &                      0.000861 \\
             100 &                  45299398 &                    61 &                      0.001347 \\
            1000 &                  45299423 &                    86 &                      0.001898 \\
\hline
\end{tabular}
	\caption{The cost of the different sampling rates for the schedule strategy compared to the LLVM solution}
	\label{table:sched_cost}
\end{table}

Table \ref{table:sched_cost} shows the cost difference between the different
sampling rates 

The cost of the schedule strategy varies slightly between sampling rates, and for the
highest rate there is 86 cycle difference comprated to the LLVM solution while the lowest
sampling rate has a 22 cycle difference. This

solution. For the lower sampling rates there are still quite a few solutions with a lower
cost than the LLVM solution so if only a few versions are necessary there is potential to
limit the constraint solver to only find those with lower cost than LLVM.

Interesting to note is that all strategies and sampling rates have found a solution with
an equally low cost. This is presumably the very first solution found; When no strategy
related constraints have been posted yet.



\section{Conclusion}

% Remember to mention the hypothesis!!!
With regards to the gadget breaking properties of the systematic approach the hypothesis
definitely holds true.

% Calculate the performance impact for carefully. Perhaps percentages in it's own plot?
To summarize the results and discussion it seems fair to draw the conclusion that enumerate
and registers, while incuring low overhead breaks very few gadgets and are thus not very
useful. The schedule strategy, however, performs very well when breaking gadgets but less
so regarding the cost.
