\section{Contraint Programming}

Constraint programming is a programming paradigm where problems are modeled as contraints
on variables. That is, a problem is solved by first identifying the variables, the domain
of each variable and the \textit{constraints} of the variables, i.e. the relation between them.

Contraint programming is used to solve combinatorial problems in a relatively efficient
manner, which makes it an excellent approach to e.g. NP-problems. Constraint programming
is somtimes also called \textit{combinatorial optimization}, and implementations are then
% https://developers.google.com/optimization/
called \textit{combinatorial optimization software suite}. This is perhaps an ever better
name for the paradigm since it describes the usage (and limitations).

% https://en.wikipedia.org/wiki/Constraint_programming#Constraint_programming_libraries_for_imperative_programming_languages
This paradigm is not limited to certain languages but are usually implemented as libraries
that implement a constraint solver. There are also a few languages that support constraint
programming natively, such as Curry.
% http://www-ps.informatik.uni-kiel.de/currywiki/_media/documentation/report.pdf
% http://www-ps.informatik.uni-kiel.de/currywiki/documentation/features
In both cases the user specifies the variables and
the constraints and the constaint solver propagates the given constraints and reduces each
variables domain until there is only one option left, and the problem is solved.

\subsection{Variables and Domains}


\subsection{Constraints}

\subsection{Constraint Solving}

\subsection{Searching}
