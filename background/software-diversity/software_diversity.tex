\section{Software Diversity}

Software diversity is a diverse field, and there is research focusing on different areas
with different goals in mind. However, what they all have in common is the that they are
exploring the potential benefits of engineering diversity within software development.
When talking about software diversity there are a few different classifications and goals
\cite[Section~1]{survey} which will be covered here.

Software diversity has a diverse (heh) set of goals ranging from fault-tolerance to re-usability
to security \cite{survey}.

\subsection{Managed Software diversity}

Managed software diversity is the notion of encouraging or controlling software diversity.

\subsubsection{Natural Diversity}

A lot of software with similar features naturally and spontaneously emerges from engineering
processes. There are a multitude of products from competing organizations that provide
the same function, for example routers, web browsers, compilers, database management systems
and firewalls. Natural diversity could also be something as simple as one program being
tunable by parameters to provide different functionality and performance \cite{survey}.

\subsubsection{Design Diversity}

Design diversity is the process of introducing diversity through the design process. For
example, \textit{N-version programming} is the approach where $N \geq 2$ functionality equivalent
programs are developed from the same specification. The point of this is that hopefully
it will result in a subset of the programs being free from bugs that another subset
might suffer from, thus creating a more fault tolerant system \cite{n-version}.

\subsection{Automated Software Diversity}

\textcite{survey} describes automated software diversity as \say{techniques for artificially
and automatically synthesizing diversity in software} \cite[\pno~8]{survey}. It is also
known as \textit{synthetic diversity} \cite{synthetic-diversity}. \say{Automated} is
referring to the fact that a human is not part of the diversification process, other than
having designed the framework \cite[Section~4]{survey}.

Automated diversity can characterize itself in different ways. A common approach is to
introduce some kind of randomness into software to break the otherwise deterministic behaviour
of most software. This can have several benefits, including security \cite{add-obfuscation}.
Randomization techniques are generally ways to either directly or indirectly create unique
execution of the same program \cite[Section~4.1]{survey}.

\subsubsection{Dynamic Randomization}

By introducing randomization at program run-time one can dynamically diversify the program
and introduce non-deterministic behaviour.

\textcite{os-randomization} randomizes the interface
between user space applications and the operating system by shuffling system call mappings,
changing library entry points and randomizing stack placement.

\textcite{mem-exploits} implements a source-to-source C-code transformer that randomizes
stack-resident variables, static data and individual functions (by introducing a level of
indirection to function calls).

\textcite{binary-stirring} introduces a technique they call \textit{stirring}, where
they accept input in the form of x86 binary code (without debug symbols, source code or
relocation information) and produce a new x86 binary file whose basics block addresses
are stirred and dynamically determined at program load-time.

Of course there are many more techniques and approaches but they will not be covered here.

\subsubsection{Static Randomization}

Static randomization creates diversity at compile time by generating several versions of the
same program that are functionally equivalent but semantically different. This can be done
in a variety of ways including but not limited to: exploiting NOP (no operation) instructions,
instruction set randomization, reversing the stack, stack frame padding,
and obfuscation (register randomization, variable reordering etc)
\cite{survey, compiler-generated-sw-div},

This thesis will focus on static, compiler generated randomization. Just as for dynamic
randomization, static randomization's current focus is security. The key insight is that
while the same source code always yields the same binary in the current climate, the compiler
takes a lot of decision based on heuristics when reaching this concluding binary (see \ref{sec:llvm}).
It is even the case that those decisions might be flawed and that a different heuristic
might have yielded a better result (see \ref{sec:unison}). Not only does this indicate
that there is room for improvements in present-day compilers, but it also a huge potential
for automated software diversity. For example, if we compile:

\lstinputlisting[label={src:gcd},caption=A C function for calculating the greatest common
divisor between two integers, language=C,tabsize=2,frame=single,breaklines=true,
showstringspaces=false, backgroundcolor=\color{lightgray}]{background/software-diversity/examples/gcd.c}

we can generate multiple different binaries simply by using multiple different compilers.
Here are two examples using gcc and clang (with -fomit-frame-pointer):

\lstinputlisting[caption=Assembly emitted by gcc for listing \ref{src:gcd},tabsize=2,
frame=single, breaklines=true,showstringspaces=false,backgroundcolor=\color{lightgray}]
{background/software-diversity/examples/gcc_gcd.s}

\lstinputlisting[caption=Assembly emitted by clang for listing \ref{src:gcd},tabsize=2,
frame=single, breaklines=true,showstringspaces=false,backgroundcolor=\color{lightgray}]
{background/software-diversity/examples/clang_gcd.s}

While this program is trivially small, the resulting binaries do differ. Generating an
ELF file for both assembly programs (with a trivial main function) on my Toshiba Portégé
Z30-A-15M running a Intel® Core™ i7-4500U-processor reveals slight differences. For example
the ELF file emitted by gcc has the entry point address at $0x540$ while the one emitted
by clang has the entry point address at $0x530$. They also vary slightly in memory and
disk layout. Whether these differences actually benefit us or not isn't as relevant right now
as much as the fact that they are two semantically different binaries that are functionally equal.

There are ways to deploy and steer this inherent diversity. \textcite{compiler-generated-sw-div}
have explored two approaches to this. One where they implement a diversification engine for
an App Store (such as Apple's App Store or Google Play Store) that automatically generates
a unique binary for every download. The other approach is to run multiple variants of the
same program at the same time in parallel and verifying that their behaviour is correct.
Both of these approaches can make use of more or less all previously mentioned diversity
techniques, both dynamic and static.

We will explore the diversity potential in integrated instruction scheduling and register
allocation by using a tool that functions by combinatorial exploration of binary programs
that potentially represents the input source code (\ref{sec:llvm} and \ref{sec:unison}).
In other words, it compiles the code by working out possible binary representations of the
program and then iteratively verifying them to find a solution.

% return oriented programming: https://www.informatik.tu-darmstadt.de/fileadmin/user_upload/Group_TRUST/PubsPDF/readactor.pdf
