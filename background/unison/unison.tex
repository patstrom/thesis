\section{Unison}
\label{sec:unison}

Unison is an open-source\footnote{https://github.com/unison-code/unison},
potentially optimal tool that performs integrated register allocation and instruction
scheduling using constraint programming. It can be used as an alternative or complement to
the algorithms currently in place by compilers such as GCC and LLVM. In particular there
already exists a driver for LLVM which accept input in the form of LLVM MIR \cite{unison-docs}.

\subsection{Constraint Programming}

Constraint programming is a programming paradigm for solving combinatorial problems.
By declaring all variables possible values and their \textit{constraints}, the relationship
between them when part of a solution, a \textit{constraint solver} can search the solution
space and find an assignment of variables to values that is consistent with the constraints.
A constraint solver effectively explores different possible combinations systematically,
by a potentially incomplete local inference (also known as \textit{constraint propagation})
or more commonly a combination of the two \cite{handbook-constraint-programming}.

Currently the only constraint solver supported by Unison is Gecode\footnote{www.gecode.org}
\cite{unison-docs}, which is a constraint solver that interleaves system search algorithms
search with constraint propagation\cite{MPG}.

When propagating constraints the Gecode solver searches all variable's domains and removes
variables that in conflict with the constraints\cite[Section~23.1]{MPG}. For example,
given two variables (and their corresponding domains) $x \in {0,1,2}$ and $y \in {0,1,2}$
and the constraint $x > y$ constraint propagation can determine that $x \in {1, 2}$ and
$y \in {0, 1}$ are the only combinations consistent with the constraint.

When constraint propagation is finished there are three possible states:

\begin{enumerate}
	\item One or more domains could be empty, proving that no solution exists.
	\item	All domains could be of size 1, indicating that there exists only one possible
		value for every variable, and thus we have found a solution.
	\item One or more variables have multiple values in their domain.
\end{enumerate}

For situation (1) and (2), either a solution is found or we have proven that a solution
does not exist (within the local search space).

In the latter situation (3) the Gecode constraint solver splits a variable's domain into
two or more subsets, creating a \texit{search tree} where each \textit{branch} represents
reducing the variable's domain to a particular subset \cite[Section~8]{MPG}. By commiting
to a branch the constraint solver can once again perform constraint propagation and repeat
the process. However, if \textit{branching} has taken place and the solver reaches
situation (1) it can go back up the tree and explore a different branch. If situation (2)
is reached it can still backtrack but with the added option of adding more constraints
based on the newly found solution. Constraining based on previously found solutions in the
search tree is done with the \textit{branch-and-bound} search engine\cite[Section~9]{MPG}
in Gecode.

\textit{Branch and bound} is an efficient strategy to find the optimal solution to a
combinatorial problem which in essence constitutes comparing potential solutions to the
currently best found solution, choosing the better of the two\cite{BaB}. We will adopt a
similar strategy, but instead of constraining solutions to be better we will require them
to be \textit{different}. In addition, no solutions will be discarded.

\subsection{Unison}


Unison models the problems of register allocation and instruction scheduling as a single
constraint-satisfaction problem and solves them simultaneously \cite{unison-docs,reg-alloc-inst-sched-uni}.

Unison's key properties are that it introduces \textit{optional copies} and
\textit{alternative temporaries} \cite{reg-alloc-inst-sched-uni}. This allows Unison to
support different register allocation decision and perform otherwise unreachable optimizations
\cite{reg-alloc-inst-sched-uni, comb-spill}.
