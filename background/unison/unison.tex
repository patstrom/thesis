\section{Unison}
\label{sec:unison}

Unison is an open-source\footnote{\url{https://github.com/unison-code/unison}},
potentially optimal tool that performs integrated register allocation and instruction
scheduling using constraint programming. It can be used as an alternative or complement to
the algorithms currently in place by compilers such as GCC and LLVM. In particular there
already exists a driver for LLVM which accept input in the form of LLVM MIR \cite{unison-docs}.

LLVM MIR is a human readable, YAML serialized format of the LLVM machine specific
representation\cite{llvm-mir-lang-ref}. The machine specific representation is used by the
LLVM code generator to emit assembly for a specific architecture \cite{welcome-to-backend}.

Unison models the problems of register allocation and instruction scheduling as a single
constraint programming problem and solves them simultaneously using a branch and bound
algorithm\cite{unison-docs,reg-alloc-inst-sched-uni,unison-src}.

To allow for further optimizations than the current LLVM code generator Unison also
introduces \textit{optional copies} and \textit{alternate temporaries}
\cite{reg-alloc-inst-sched-uni}. These two techniques increases the number of possible
executables that can be generated by introducing more combinations into the constraint
satisfaction problem. While primarily used for code optimization they can also be used for
diversity.

The implementation strategy for these techniques is to create an abstract \textit{operation},
which can potentially be implemented by several instructions, or not at all. In addition,
the \textit{operands} are considered constraint variables that can be connected to one out
of a set of temporaries \cite{unison-docs}, or not be connected at all. By creating a
level of indirection between operations and instructions as well as their corresponding
operands and temporaries the problem can more easily be modeled as a constraint
satisfaction problem where operations and operand are the variables with instructions and
temporaries as domains, respectively.

With Unison we have control over every decision made related to instruction scheduling and
register allocation, and more importantly we can base decisions on previously found solutions.
Unison works on the function level and a solution in this case refers to an instruction
schedule and register allocation for the given function that is executable on the target
architecture.
