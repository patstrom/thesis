\begin{abstract}
	Unison is a tool that combines instruction scheduling and register allocation as a
	single combinatorial problem and solves it using constraint programming, which is a
	programming paradigm for systematically solving combinatorial problems.

	Automated software diversity is the process of automatically providing diverse executables
	in an effort to break so called \textit{gadgets}, which are short instruction sequences
	that together make up an attack vector. Attacks that utilize gadgets rely heavily on the
	arrangement of the code in the executable. By providing a population of executables with
	equivalent functionality but different arrangements an adversary must construct a unique
	payload for each executable. The idea is to mount a proactive defense against
	adversaries and limit the reusability of each constructed payload.

	The results when using Unison to systematically generate diverse executables show that
	the number of possible pairwise distinct executables is often larger than 1000000, even
	for small functions (less than ten instructions). Using Unison to force the executables
	to differ in a particular way is simple to implement, only a handful lines of code. One
	strategy evaluated in the experiment resulted in that the most frequent gadget only
	appeared in 24\% of versions, and 82\% of the gadgets only appeared in one program
	version each.  However, future work is required before anything consumer oriented can be
	evaluated, in part because Unison does not support the x86 architecture.
\end{abstract}
