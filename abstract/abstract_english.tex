\begin{abstract}
	Unison is a tool that combines instruction scheduling and register allocation as a
	single combinatorial problem and solves it using constraint programming, which is a
	programming paradigm for systematically solving combinatorial problems.

	Automated software diversity is the process of automatically providing diverse executables
	in an effort to break so called \textit{gadgets}, which are short instruction sequences
	that make up an attack vector. Attacks that utilize gadgets rely heavily on the
	arrangement of the executable, and by providing a population of programs with equivalent
	functionality but different arrangements (in an attempt to break gadgets) an adversary
	must construct a unique payload for each target.

	When using Unison to systematically generate diverse executables I found that the size
	of the population is often larger than 1000000, even for small functions (<10 instructions).
	Forcing the executables to differ in a certain way is simple to implement, only a
	handful lines of code, and one strategy for diversity resulted in that 82\% of gadgets
	only appeared in one version, and the most frequent gadget only appeared in 24\% of
	versions. However, future work is required before anything practical can be evaluated,
	in part because Unison does not support the x86 architecture.

\end{abstract}
