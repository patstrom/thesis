\begin{otherlanguage}{swedish}
  \begin{abstract}
		Unison är ett verktyg som kombinerar instruktionsschemaläggning och registerallokering
		som ett enat kombinatoriskt problem och löser det med villkorsprogrammering, vilket
		är en programmeringsparadigm för att systematiskt lösa kombinatoriska problem.

		Automatiserad mjukvarumångfald innefattar att generera varierade exekverbara filer i
		ansats att bryta så kallade \textit{gadgets}, vilket är korta instruktionssekvenser
		som utgör en attackvektor. Attacker som utnyttjar gadgets förlitar sig mycket på
		filens interna struktur. Genom att generera en population av exekverbara filer med
		ekvivalent funktionalitet men strukturella skillnader så måste en motspelare
		konstruera en unik attack för varje exekverbara fil.
		
		Att använda Unison för att systematisk generera varierade exekverbara filer visar att
		antalet möjliga, parvis distinkta exekverbara filer är ofta större än 1000000, även
		för små funktioner (<10 instruktioner). Att använda Unison för att tvinga de
		exekverbara filerna att skilja sig på specifika sätt är enkelt att implementera, bara
		en handfull rader kod.  En utvärderad strategi i experimentet resulterade i att den
		gadget med högst frekvens återfanns i 24\% av alla programversioner, och 82\% av alla
		gadgets återfanns i endast en programversion var. Dock så krävs mer arbete innan någon
		konsumentorienterat kan utvärderas, delvis för att Unison inte stödjer
		processorarkitekturen x86.
	\end{abstract}
\end{otherlanguage}
