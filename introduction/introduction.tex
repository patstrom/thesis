\chapter{Introduction}

Work in progress

\section{Problem Statement}

It might feel intuitive to provide software diversity by somehow randomizing a part of an
executable but this approach comes with a few problems. When generating the executables
you cannot make any guarantees as to whether it actually worked nor that all possible executables
are generated. For example an instruction sequence of length 3 only has 6 permutations, so
when generating 7 binaries \textit{at least} two will be equal. If generated by a chance-based
algorithm then all 7 could we equal for all the user knows. While there are many more ways
to diversify an executable the concept extends.

By instead systematically searching the space of different ways to transform the program
we can generate the entire \textit{diversity space} (all pair-wise sufficiently different
executables of the same program). In the example above that would entail deterministically
generating the 6 different permutations. Not only would this provide is with the largest
number of diverse executables, we could also prove that it is indeed the largest set.

The primary tool that will be used to achieve this, called Unison, (see \ref{sec:unison}),
is mainly a code generator that does integrated instruction scheduling and register
allocation. This limits our code transformations to what Unison is capable of.

\subsection{Hypothesis}

Systematically generating diverse executables with Unison is at least as effective at
thwarting ROP attacks as basing the diversification process on randomization.
