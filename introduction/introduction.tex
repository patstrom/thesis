\chapter{Introduction}

Work in progress

\section{Problem Statement}

It might feel intuitive to provide software diversity by somehow randomizing a part of an
executable but this approach comes with a few problems. When generating the executables
you cannot make any guarantees as to whether it actually worked nor that all possible executables
are generated. For example an instruction sequence of length 3 only has 6 permutations, so
when generating 7 binaries \textit{at least} two will be equal. If generated by a chance-
based algorithm then all 7 could we equal for all the user knows. While there are many
more ways to diversify an executable the concept extends.

By instead systemtically searching the space of different ways to transform the program
we can generate the entire \textit{diversity space} (all pair-wise sufficiently different
executable of the same program). We can thus generate the largest set of executables that
satisfy our demands and prove that it is indeed the largest set.

\subsection{Hypothesis}

Systematically generating diverse executables is at least as effective at thwarting ROP
attacks as basing the diversification process on randomization.
