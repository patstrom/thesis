\begin{frame}
	\frametitle{Unison}
	
	Combined instruction scheduling and register allocation.

	\vspace{0.5cm}

	Based on combinatorial optimization (constraint programming).

\end{frame}

\begin{frame}
	\frametitle{LSSA}
	\textit{Linear Static Single Assignment}
	
	All variables are local to basic blocks and made live-in/live-out as necessary.

	\vspace{0.5cm}

	Makes it possible to model register allocation as a rectangle packing problem where
	each rectangle represents a live-range.

\end{frame}

\begin{frame}
	\frametitle{Features}
	\begin{itemize}
		\item Optional copies

			Introduce optional copies that can be implemented by alternate instructions

		\item Alternative Temporaries

			Introduce possibility of alternate temporaries that can break data dependencies
	\end{itemize}
\end{frame}

\begin{frame}
	\begin{columns}
		\begin{column}{0.5\textwidth}
			LLVM MIR ( Hexagon)

			\lstinputlisting[basicstyle=\tiny,tabsize=2,frame=single,breaklines=true,showstringspaces=false]
			{../background/llvm/examples/factorial.mir}
		\end{column}

		\begin{column}{0.5\textwidth}
			Unison IR

			\lstinputlisting[basicstyle=\tiny,tabsize=2,frame=single,breaklines=true,showstringspaces=false]
			{../background/unison/examples/factorial.uni}
		\end{column}
	\end{columns}
\end{frame}

\begin{frame}
	\begin{columns}
		\begin{column}{0.5\textwidth}
			Unison IR
			\lstinputlisting[basicstyle=\tiny,tabsize=2,frame=single,breaklines=true,showstringspaces=false]
			{../background/unison/examples/factorial.uni}
		\end{column}

		\begin{column}{0.5\textwidth}
			Linearized
			\lstinputlisting[basicstyle=\tiny,tabsize=2,frame=single,breaklines=true,showstringspaces=false]
			{../background/unison/examples/factorial.lssa.uni}
		\end{column}
	\end{columns}
\end{frame}

\begin{frame}
	\begin{columns}
		\begin{column}{0.5\textwidth}
			Linearized
			\lstinputlisting[firstline=8,lastline=14,basicstyle=\tiny,tabsize=2,frame=single,breaklines=true,showstringspaces=false]
			{../background/unison/examples/factorial.lssa.uni}
		\end{column}

		\begin{column}{0.5\textwidth}
			Extented (optional copy)
			\lstinputlisting[firstline=13,lastline=29,basicstyle=\tiny,tabsize=2,frame=single,breaklines=true,showstringspaces=false]
			{../background/unison/examples/factorial.ext.uni}
		\end{column}
	\end{columns}
\end{frame}

\begin{frame}
	Augmented (alternate temporaries) - Snippet of loop body
		\lstinputlisting[firstline=15,lastline=33,basicstyle=\tiny,tabsize=2,frame=single,breaklines=true,showstringspaces=false]
		{../background/unison/examples/factorial.alt.uni}
\end{frame}

\begin{frame}
	\frametitle{Solving}

	Also takes into account processor specific features and resources.

	\vspace{0.5cm}

	Such as which instructions can implement an operation, latencies of these instructions,
	register capacity and packing capability.

\end{frame}
