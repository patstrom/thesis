\begin{frame}
	\frametitle{Unison}
	
	Combined instruction scheduling and register allocation

	\vspace{0.5cm}

	Models the problem for a constraint solver \cite{unison-docs}

	\vspace{0.5cm}

	Introduces alternatives, creating more possible combinations

\end{frame}

\begin{frame}
	\frametitle{LLVM Machine Specific Representation}
	
	The data structure(s) used by the LLVM code generator to emit assembly \cite{llvm-mir-lang-ref}

	\vspace{0.5cm}

	Not close enough to the actual machine code to search for gadgets in
	
\end{frame}

\begin{frame}
	\frametitle{Diversifying}

	\textcite{large-scale-automated} tried randomizing instruction schedule and inserting no-ops at random places in the code.

	\vspace{0.5cm}

	Proved to be successful

	\vspace{0.5cm}

	Unison can do both. Perhaps better.

\end{frame}

\begin{frame}
	\frametitle{Hypothesis}

	Systematically generating diverse executables with Unison is at least as effective at
	thwarting ROP attacks as basing the diversification process on randomization.
\end{frame}
