\begin{frame}
	\frametitle{Code re-use attacks}

	Divert the NX bit.

	\vspace{0.5cm}

	Hijack control flow.

	\vspace{0.5cm}

	Execute already loaded code in an unintended manner.

\end{frame}

\begin{frame}[fragile]
	\frametitle{Return Oriented Programming}

	Short instruction sequences, dubbed \textit{gadgets}, ending in a ret instruction.

	\begin{columns}
		\begin{column}{0.5\textwidth}
			\begin{lstlisting}[frame=single,language={[x86masm]Assembler}]
pop %ecx
pop %edx
ret
			\end{lstlisting}
		\end{column}

		\begin{column}{0.5\textwidth}
			\begin{lstlisting}[frame=single,language={[x86masm]Assembler}]
xor %eax, %eax
ret
			\end{lstlisting}
		\end{column}
	\end{columns}

	\vspace{0.5cm}

	Can be chained together to perform arbitrary operations.

	\vspace{0.5cm}

	Introduced on x86, where they are frequent.

\end{frame}

\begin{frame}
	\frametitle{Defenses}

	Payload relies on the exact addresses of the gadgets.

	\vspace{0.5cm}

	Compilers make many decisions, not all are the only valid one.

	\vspace{0.5cm}

	Population where a payload only works on one executable.

	\vspace{0.5cm}

	Many solutions today introduce chance somewhere in the process.

	\vspace{0.5cm}

	Systematically explore the combinations of decisions.

	\begin{itemize}
		\item Guarantees about population
		\item Correctness of transformation
		\item Fine-grained control of process
	\end{itemize}

\end{frame}
