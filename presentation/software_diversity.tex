\begin{frame}
	\frametitle{Automated Software Diversity}

	Using some kind of tool to generate several semantically different but functionally
	equal executables.

\end{frame}

\begin{frame}
	\frametitle{Some Approaches}

	\begin{itemize}
		\item \textcite{os-randomization}
randomizes the interface between user space applications and the operating system,
		\item \textcite{mem-exploits} introduces a level of indirection to function call to
			randomize static data at the C source code level
		\item \textcite{binary-stirring} dynamically determine the addresses of basic blocks
			at load time,
		\item \textcite{librando} implements a library to transparently diversify code compiled
			and executed by JIT-compilers.
	\end{itemize}

\end{frame}

\begin{frame}
	\frametitle{Why Automated Software Diversity?}

	\begin{itemize}
		\item Security $\leftarrow$
		\item Obfuscation
		\item	Error-handling
	\end{itemize}

\end{frame}


\begin{frame}
	\frametitle{Return Oriented Programming}
		\begin{itemize}
			\item Code re-use attack
			\item extends \textit{return-into-libc}
			\item Arbitrary code execution
		\end{itemize}
\end{frame}

\begin{frame}
	\frametitle{Return-into libc}

	\begin{itemize}
		\item Hijack control flow
		\item Call unintended function that is already loaded, e.g system()
		\item Setup parameters with short code sequences such as "pop reg; ret"
	\end{itemize}
\end{frame}

\begin{frame}
	\frametitle{ROP}

	Only use these short code sequences (that typically end with a ret)

	\vspace{0.5cm}

	\textcite{rop} (who introduced the concept) showed you can execute arbitrary code, e.g
	open a shell.
\end{frame}
