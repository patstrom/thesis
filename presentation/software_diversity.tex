\begin{frame}
	\frametitle{Automated Software Diversity}

	Using some kind of tool to generate several semantically different but functionally
	equal executables or runtimes.

	\vspace{0.5cm}

	Specifically we will generate multiple executables at compile time.

\end{frame}

\begin{frame}
	\frametitle{Some Approaches}

	\begin{itemize}
		\item \textcite{os-randomization}
randomizes the interface between user space applications and the operating system,
		\item \textcite{mem-exploits} introduces a level of indirection to function call to
			randomize static data at the C source code level
		\item \textcite{binary-stirring} dynamically determines the addresses of basic blocks
			at load time,
		\item \textcite{librando} implements a library to transparently diversify code compiled
			and executed by JIT-compilers.
	\end{itemize}

\end{frame}

\begin{frame}
	\frametitle{Why Automated Software Diversity?}

	\begin{itemize}
		\item Security
		\item Obfuscation
		\item	Error-handling
	\end{itemize}

\end{frame}


\begin{frame}
	\frametitle{Return Oriented Programming}
		\begin{itemize}
			\item Code re-use attack
			\item extends \textit{return-into-libc}
			\item Arbitrary code execution without code injection
		\end{itemize}
\end{frame}

\begin{frame}
	\frametitle{Return-into libc}

	\begin{itemize}
		\item $W \oplus X$ defense
		\item Somehow hijack control flow
		\item Call functions that are already loaded (e.g system) in an unintended manner
		\item Setup parameters / chain function call with short code sequences such as "pop reg; ret"
	\end{itemize}
\end{frame}

\begin{frame}[fragile]
	\frametitle{ROP}

	Only use these short code sequences (that typically end with a ret).

	\vspace{0.5cm}

	\begin{columns}
		\begin{column}{0.5\textwidth}
			\begin{lstlisting}[tabsize=2,frame=single,language={[x86masm]Assembler}]
pop %edi
ret
			\end{lstlisting}
		\end{column}

		\begin{column}{0.5\textwidth}
			\begin{lstlisting}[tabsize=2,frame=single,language={[x86masm]Assembler}]
movl 64(%eax), %eax
ret
			\end{lstlisting}
		\end{column}
	\end{columns}

	\vspace{0.5cm}

	\textcite{rop} (who introduced the concept) showed you can execute arbitrary code, e.g
	open a shell, and that they are quite common in x86-64.

\end{frame}

\begin{frame}
	\frametitle{Hidden Gadgets}
	Not explicitly placed there by the compiler

	\vspace{0.5cm}

	\textcite{rop} used an analogy in the English language

	\vspace{0.5cm}

	Both the word "dress" and "head" and be found in "the address"

	\vspace{0.5cm}

	For architectures with varying instruction sizes this can mean many more gadgets
	than apparent from the assembly (such as x86-64)

\end{frame}

\begin{frame}
	\frametitle{Defenses}

	Introduce randomness somewhere so that a payload is only useful once

	\vspace{0.5cm}

	Problems:
	\begin{itemize}
		\item No fine-grained control
		\item No guarantees
	\end{itemize}

	\vspace{0.5cm}

	E.g a sequence of 3 instructions only has 6 permutations. Generating 7 would yield 
	\textit{at least} two that are equal. With randomization all could be.
	\vspace{0.5cm}

	Try to deterministically generate all of them instead
\end{frame}
