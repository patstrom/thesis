\begin{frame}
	\frametitle{Constraint Programming}

	Programming paradigm for solving combinatorial problems.

	\vspace{0.5cm}

	Declare variables, their potential values (\textit{domains}) and their relationships when
	part of a solution (\textit{constraints})

\end{frame}

\begin{frame}
	\frametitle{Solving}

	Utilize a \textit{constraint solver} to explore different combinations of variable to
	value mappings

	\vspace{0.5cm}

	Local inference (aka \textit{constraint propagation}) interleaved with systematic search \cite{handbook-constraint-programming}

\end{frame}

\begin{frame}
	\frametitle{Constraint Propagation}
	Locally determine what values are incompatible with the constraints

	\vspace{0.5cm}

	Variables $x,y \in \{0,1,2\}$ and the constraint $x > y$

	\vspace{0.5cm}

	$x \in \{1,2\}$ and $y \in \{0,1\}$
\end{frame}

\begin{frame}
	\frametitle{Search}

	When constraint propagation is finished there are three possible states:

	\vspace{0.5cm}

	\begin{enumerate}
		\item One or more domains could be empty, proving that no solution exists.
		\item	All domains could be of size 1, indicating that there exists only one possible value for every variable, and thus we have found a solution.
		\item One or more variables have multiple values in their domain.
	\end{enumerate}

\end{frame}

\begin{frame}
	\frametitle{Branching}

	\begin{enumerate}
		\item Choose a variable and construct a number of subsets based on it's domain
		\item Reduce the variable's domain to one subset, creating a \textit{branch} in a \textit{search tree}
		\item Propagate and repeat
	\end{enumerate}

	\vspace{0.5cm}

	Now when we reach (failure) or a (solution) we can still continue.

\end{frame}

\begin{frame}
	\frametitle{Branch and Bound}

	Compare new solutions to the currently best found and choose the better \cite{BaB}

	\vspace{0.5cm}

	Primarily used for optimization. Can it be used to diversify?

\end{frame}
