\chapter{Diversification with Unison}

In this chapter the relevant parts of the Unison model, the modification to the models that
will be implemented and the experimental protocol will be presented.

The first section, \textit{Automatic Diversity Synthesis with Unison} briefly presents the
Unison model and how to implement our extension of it. The \textit{disUnison} section
describes how we extend the model to provide software diversity, and the
\textit{Experimental Protocol} section describes how the experiment will be conducted.

\section{Automatic Diversity Synthesis with Unison}
\label{sec:unison-model}

In its current state Unison accepts input in the form of the LLVM MIR (see Section
\ref{sec:unison}) of a single function \cite{unison-docs} and outputs LLVM MIR of the same
function with allocated registers and scheduled instructions. It is then the job of
\textit{llc}, the LLVM static compiler to emit architecture specific assembly, which can
be passed through a native assembler and linker to generate an executable \cite{llvm-llc}.

As described in Section \ref{sec:unison} the problem of integrated register allocation and
instruction scheduling in Unison is modeled around \textit{operations} and \textit{operands}.
The problem consists of around 20 variables (see \ref{sec:constraint}) in total so only the
ones relevant for the experiment will be presented. These variables describe how the
operations and their operands relate to the actual instructions, temporaries, registers
and issue cycles.

The only constraint solver currently supported by Unison is Gecode \cite{unison-docs}. A
branch and bound search can be implemented in Gecode by using a branch and bound search
engine and overriding the virtual member function \textit{constrain()} of the model. The
\textit{constrain()} function is invoked by the search engine whenever a solution is found
and takes the most recently found solution as an argument. The idea is to post constraints
on the next solution based on the previous solution. These constraints accumulate so
that all future solutions will be affected by all already found solutions.

As mentioned, our intention is to replace the current \textit{constrain()} function of
Unison with one that instead posts constraints to ensure that future solutions are
\textit{different} in some regard.

\subsection{Cost}
\label{sec:cost}

An important variable in the Unison model is \textit{cost}. It is the deciding factor
of whether a solution is better than another during the branch and bound search. It is a
sum of the estimated cost of each basic block, weighted by the estimated execution
frequencies \cite{unison-docs}. Cost can be either cycles or code size depending on if the
optimization goal is speed or size, respectively. That is the \textit{constrain()} function
of the Unison model post the constraint that for future combinations to be considered
solutions, in addition to the original constraints, the cost must be less than the cost of
the previously found solution. Given enough execution time, Unison will thus find the
optimal solution with regards to the \textit{cost} variable.

\subsection{Target Architecture}
\label{sec:arch}

Unison does not currently support the x86 or x86-64 architecture. Only ARM, Hexagon and MIPS
are supported \cite{unison-src}. None of the supported architectures are generally considered
when testing automated software diversity, but for the purposes of evaluating the use of
a systematic approach the supported architectures offers a glimpse at the potential. For
the experiment the code will be compiled for the Hexagon architecture.

Hexagon functions very differently from x86 but for our purposes targeting Hexagon will
still hint at the gadget-breaking potential of the systematic approach. After all,
breaking gadgets is about shifting, adding, removing or otherwise modifying \textit{any}
instruction in the program, and since the diversification strategies (see Section
\ref{sec:disUnison}) are applied globally they are supposedly equally effective regardless
of the placement or structure of the instruction.

\section{disUnison}
\label{sec:disUnison}

The implementation that makes up the following strategies that are used in the experiment
is henceforth referred to as disUnison (from the word disunity). It is a variation of the
original Unison model, where the key difference is that the behaviour of the branch and
bound search is modified. The bulk of the Unison model, that ensures functional code is
emitted, is still present in the disUnison model.

\subsection{Strategies}
\label{sec:strategies}

For the purposes of this thesis we define three strategies to be evaluated and compared.
The goal of each strategy is to provide a population of as diverse versions of an
executable as possible while incurring as little overhead as possible. The chosen
strategies are \textit{enumeration}, \textit{instruction schedule} and
\textit{register allocation}, and the motivation for each is presented in each respective
subsection.

\subsubsection{Enumeration}

The name of this strategy comes from the fact that we are only concerned that the solutions
are different, not how they differ. During search the Gecode search engine will never
explore the same combination twice, and thus never generate two equal solutions. The
strategy is thus to not post any constraints at all and let the solver generate all
possible combinations. We \textit{enumerate} the solutions.

Unison can differ the solutions in four main ways:

\begin{itemize}
	\item The order of the operations is different
	\item Operands are connected to different registers
	\item Execute a copy using a different instruction (or not at all)
	\item Split live-ranges and spill temporaries differently
\end{itemize}

The results of this strategy serve as a baseline for the program. It is literally what
happens if we do nothing.

\subsubsection{Registers}

The strategy to diversify the register allocation of the resulting binaries is an attractive
one due to causing no run-time overhead. Consequently, if it introduces significant diversity
it is an excellent candidate. In addition, as register allocation is one of Unison's
primary purpose it feels like a natural strategy to explore.

There are two variables and one set that are of concern when diversifying the register
allocation. Their description from the Unison documentation are as follows:

\vspace{0.2cm}

\noindent\makebox[\textwidth]{
	\begin{tabular}{c|lr}
		\textbf{Type} & \textbf{Name} & \textbf{Description} \\ \hline
		\textbf{Set} & $P$ & The set of all operands in the program \\ \hline
		\multirow{2}{*}{\textbf{Variable}}
			& $x_p \in \{0, 1\}$ & whether operand \textit{p} is connected (0 is false and 1 is true) \\
			& $ry_p \in \mathbb{N}_0$ & register to which operand \textit{p} is assigned \\
	\end{tabular}
}

\vspace{0.2cm}

In pseudo mathematical notation we want to post the constraint:

\vspace{0.2cm}
\noindent\makebox[\textwidth]{
	$\neg (\bigwedge\limits_{p \in P} ( (x_p = prev.x_p) \land (ry_p = prev.ry_p) ))$
}
\vspace{0.2cm}

Very important to note is that the variables preceded by \textit{prev.} (i.e $prev.x_p$ and
$prev.ry_p$) represents an actual value. More precisely the value that is part of the previous
solution. For example $prev.xy_p$ represents the register to which operand $p$ is assigned
in the previous solution. The variables not preceded by \textit{prev.} are variables in
the constraint programming sense and their corresponding domains make up the remaining
combinations to explore.

In words; We disallow the exact same combination of connected (used) operands and
operand to register mapping.

The $ry_p$ variable of the model is actually an auxiliary variable that combines the
$r_t$ and $y_p$ variables. The $r_t$ variable represents which register temporary $t$ is
assigned, and $y_p$ represents which temporary is connected to operand $p$. In other words,
$ry_p$ is implemented as $r(y(p))$. This distinction is important during search, and in
particular when branching.

\subsubsection{Instruction Schedule}
\label{sec:schedule}

Given how Unison functions diversifying the instruction schedule is an exciting strategy.
As mentioned in Section \ref{sec:unison} Unison explores optional copies. In practice this
means that during pre-processing optional \textit{operations} are inserted so not only
does Unison decide on the order the instructions are executed, but in a limited capacity
Unison also inserts instructions (or deems instructions unnecessary). For the purposes
of breaking gadgets shifting instructions can help immensely as an adversary is reliant
on the exact addresses of the gadgets.

There are two variables and one set of interest for this strategy. In the Unison
documentation they are described as follow:

\vspace{0.2cm}

\noindent\makebox[\textwidth]{
	\begin{tabular}{c|lr}
		\textbf{Type} & \textbf{Name} & \textbf{Description} \\ \hline
		\textbf{Set} & $O$ & The set of all operations in the program \\ \hline
		\multirow{2}{*}{\textbf{Variable}}
			& $a_o \in \{0, 1\}$ & whether operation \textit{o} is active (0 is false and 1 is true) \\
			& $c_o \in \mathbb{N}_0$ & issue cycle of operation \textit{o} \\
	\end{tabular}
}

\vspace{0.2cm}

In order to disallow the same set of active operations combined with the same issue cycles
we post the constraint: 

\vspace{0.2cm}
\noindent\makebox[\textwidth]{
	$\neg ( (\bigwedge\limits_{o \in O} (a_o = prev.a_o)) \land (\bigwedge\limits_{m \in \{O | prev.a_m = 1\}} (c_m = prev.c_m)) )$
}
\vspace{0.2cm}

Just as for the registers strategy constraint, the variables preceded by \textit{prev.}
represents the actual value that is part of the previous solution, whereas the ones not
preceded by \textit{prev.} are the constraint programming variables used when searching for
future solutions.

The constraint described with words is that we disallow the exact same combination of
active operations and their corresponding issue cycle. We are only concerned about the
issue cycle of the previous solution's active operations, but we do take into account that
future solutions can have the same instructions issued at the same cycles if it has fewer
or more active operations.

As mentioned in Section \ref{sec:unison}, some operations can be implemented by multiple
instructions. For the purpose of breaking gadgets we do not want to allow functionally
equivalent instructions. While they would indeed make the sequence of bytes differ, the
functionality would be the same and the gadget would survive.


\subsection{Sampling Rate}
\label{sec:sampling_rate}

When exploring combinations with a constraint solver similar solutions are found close to
each-other (in-time). A factor in diversification might be to exploit this property
alongside the diversification strategy. Certain strategies might be favorable when
considering execution time but not particularly good at breaking gadgets. However, if
e.g 100 solutions are discarded between every emitted executable perhaps more gadgets are
broken.

In the original Unison model the cost variable is used both for branching and during the
branch and bound process. In the disUnison model it is still used for branching. Lower
cost combinations are explores first. However, future solutions are not bound to have
lower cost. Discarding solutions can thus impact performance in a negative way in the
sense that the later versions might have a higher cost, resulting in a wider cost
distribution across all program versions.

Sampling rates of 1, 10, 100 and 1000 will be evaluated for every strategy, where a
sampling rate of 100 means that every 100th solution is kept. Generating 1000 versions
at a sampling rate of 100 would mean that 100000 solutions are explored, 1000 are emitted
and 99000 are discarded.

The number of possible combinations is of course not limitless. 1000 version at a sampling
rate of 1000 means that there needs to be at least 1000000 possible solutions. Unison works
at the function level and for every function there might not be 1000000 possible versions.

Total number of possible combination would be an interesting metric to evaluate, unfortunately
it varies widely between functions and for some it might require days of search. Empirically,
most functions in the suite to be used do have 1000000 versions, so 1000 versions appears
to be a good number of versions to generate.

For those function where 1000 versions cannot be generated for the given strategy and
sampling rate the ones that have been generated will be re-used so that 1000 program
versions can still be generated. More information about these functions can be seen in
appendix \ref{appendix:function_names}.

\subsection{Branching Strategy}
\label{sec:branch_strategy}

The disUnison models uses the same branching strategy as the original Unison model. When
the search engine reaches a state where branching is necessary the first decision is to
assign the \textit{cost} (see Section \ref{sec:cost}) variable to its lowest possible
value. If the \textit{cost} variable is already assigned, the branching is done as
follows, in the order listed:

\begin{enumerate}
	\item assign the active operations. (the $a_o$ variable)
	\item assign which instruction should implement each operation.
	\item assign which temporary is connected to each operand. (the $y_p$ variable)
	\item assign which cycle each operation is issued. (the $c_o$ variable)
	\item assign which register is assigned to which temporary. (the $r_t$ variable)
\end{enumerate}

\chapter{Experimental Setup}

In order to evaluate both the code generator and the generated code for each strategy a
population of programs for each strategy is required. In this section the data set, the
evaluation metrics and the process for generating the program populations will be
presented.

The experiment will be carried out on a computer running a 4-core Intel(R) Core(TM)
i7-4500U CPU @ 1.80GHz and with 8 gigabytes of memory.

\subsection{Data Set}

The data set to be used is part of the Unison test suite for the Hexagon architecture. In
total 23 functions will be used, each of which is from a benchmark in the SPEC2006\footnote{https://www.spec.org/cpu2006/ (visited on 21/06/2018)}
suite. These function will together make up a \textit{program}. Since they do not make up
a complete executable they cannot be linked nor executed. Linking will instead be simulated
by placing them in the same order every time to ensure a fair comparison between
strategies and sampling rates.

As mentioned in Section \ref{sec:unison-model} Unison works on the function level, and so
does disUnison. 1000 versions of each function will be generated and labeled from 0 through
999. Version 0 of each function will make up program version 0, version 1 of each function
will make up program version 1 and so forth. That is, one program version consists of 23
functions, and there will be a total of 1000 program versions for each strategy and
sampling rate, yielding a total of 12 programs with 1000 versions each.

\subsection{Metrics}
\label{sec:metrics}

The metrics to be evaluated are surviving gadgets, \textit{cost} (See \ref{sec:cost}),
both speed and size, and the execution time of the code generator.

The \textit{cost} metric is calculated by a tool called \textit{uni analyze}, which is
part of the Unison toolchain. It accepts the LLVM MIR of a function as input and outputs
the estimated cost for each optimization goal as described in Section \ref{sec:cost}. The
cost of a program version will be calculated as the geometric mean of the cost for each
function version that makes up the program version in question.

For the experiment the optimization goal will be speed, and thus solutions that are
estimated to execute faster are generated first. However, both cost in speed and cost in
size will be presented in the results section.

Surviving gadgets will be calculated as the ratio of which each gadget appears among the
population of program versions. All gadgets present in any of the 1000 program versions
will be enumerated. The ratio is calculated as the number of occurrences of every unique
gadget divided by the number of versions, which in this case is 1000. In other words a
ratio of 100\% means that the gadget appears in all version and the strategy was not
effective at breaking that gadget. Similarly a very low ratio (close to 0\%) would mean
that the strategy was effective as the gadget only appears in a small number of the
programs.

\subsection{Generation Process}

Our process to generate our test data is as follows:

\begin{enumerate}
	\item For every function generate 1000 versions with "speed" as the optimization goal.
	\item Version 0 of every function will make up program 0, version 1 will make up program
		1 and so forth.
	\item For every program find all gadgets and its cost.
	\item Calculate our metrics.
	\item Repeat for all strategies.
		\begin{itemize}
			\item Enumerate.
			\item Registers.
			\item Schedule.
		\end{itemize}
	\item Repeat for all sampling rates.
		\begin{itemize}
			\item 1.
			\item 10.
			\item 100.
			\item 1000.
		\end{itemize}
\end{enumerate}

In other words, for every combination of strategy and sampling rate (12 in total) 1000
different versions of each of the 23 functions will be generated. These function versions
will then be combined to yield 1000 different program versions.
