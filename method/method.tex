\chapter{Exploring the Unison Diversity Space}

%Since Unison uses a branch and bound strategy to find the optimal solution there are a lot
%of solutions that we simply discard. In this section we will focus on exploring these
%solutions. How many we can generate, how diverse they are as-is and what applications this
%might have. In further chapters we will explore how we can modify Unison to transform the code to
%provide diversity.

%\section{Potential Applications}
%
%I'd like to start this chapther with the potential applications of something like this.
%If we one day replace the current LLVM instruction scheduler and register allocator in our
%toolchain with Unison all of this potential diversity will be given to us for free, so there
%is no real point in \textit{not} exploring and hopefully applying it.
%
%One could imagine an assembly line of embedded devices, all running the same firmware, where
%a unique binary of the firmware is compiled for every new device as it is ready to be loaded.
%This would

%Does all binaries have to be within a (e.g) 5-cycle cost of each other? Can we just spit something
%out every X seconds and assume it's different enough? Can we choose the variation depending
%on how many binaries the user wants?

%\section{Program Set}
%
%Throughout this chapter we will be working with a set of benchmarks called something
%Roberto knows.
%
%I will now commence explaining why these are appropriate in a way that doesn't make it
%obvious that it's because they are used as tests in Unison.
%
%\section{The Natural Space}
%
%The first step is to look at what is available. I.e. what diversity can be identify in the
%unmodified Unison search space. Unison upper bound on the optimizing factor is often a
%gross over-estimation, thus some solutions will be several orders of magnitude worse than
%the optimal solution. Since we want to minimize the run-time effect on our diversified
%program we only want to consider programs within a reasonable distance of the optimal
%solution. For example if the optimal solution executes in 7 cycles then a solution that
%executes in 52 cycles is not relevant for any application.
%
%\subsubsection{Differentiating Factors}
%
%When comparing two different solution there needs to be a way to quantify how different
%they are. This is a strict requirement to add diversify as a cost function in Unison and
%implementing a "diversity" optimization goal.
%
%A good starting point for this to look at the diversification techniques presented by
%\textcite[Section~4.4]{compiler-generated-sw-div} and \textcite{survey} and attempt to
%quantify how many of these are present and in what capacity.
%
%\subsubsection{More Than one Optimal Solution}
%
%Something to also explore is how many optimal solutions there are. During Unison's branch
%and bound strategy subsequent solutions has to be \textit{better} to even be considered,
%but how many \textit{equally good} solutions can find? And how different are they?
