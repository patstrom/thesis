\chapter{Diversification with Unison}


The two strategies explored by \textcite{large-scale-automated} proved to be effective in
the sense that very few gadgets survived between the emitted binaries. Specifically, the
strategies consits of randoimizing the instruction schedule or inserting no-op instructions
randomly in the emitted assembly.

Summarized, the prodecure \textcite{large-scale-automated} used to evaluate the individual
strategies was to generate 25 different versions for each transformation and testing their
performance, frequency of surviving gadgets and file size compared to the original. For
our purposes the instruction schedule transformation is the most interesting, as it relates
more directly to Unison.

The presented result is that randomizing the instruction schedule resulted in an 9\%
slowdown and that it removes on average 95\% of gadgets \textit{with respect to the original executable}.
The file size test for the schedule transformation are described as "inconclusive" but
also that "randomized instruction scheduling has only a negligible effect of file size".
\footnote{More comprehensive tests were done on other transformations, including pairwise
testing of surviving gadgets.}

The problem with the above approach is first and foremost that the amount of generated
binaries is very small. Presumably it will lose effectiveness after some number of
generated binaries, and at some point the pidgeonhole principle will come into effect.
What is not explored is how many different versions can be generated and how different do
they have to be to achieve similar results. In this chapter we aim to explore this.

\section{Gecode Constraint Solver}

Gecode is implemented such that a problem is modeled in a class inheriting from the
\textit{Space} base class. It is in this derived class variables are defined and constraints
are posted. When a solution is found by the Gecode branch and bound engine the virtual
member function \textit{constrain(b)} is called, where \textit{b} is the latest solution.
This function is expected to post constraints on future solutions based on the latest
solution. These constraints accumulate so posting constraints based on the previous solution
is sufficient for diversity.

Each strategy is implemented by constraints that, when satisified, ensures the
generated solutions differ in the correct manner. These constraints are posted in the
\textit{constrain()} function. In general the constraints will make sure that future
solutions does not contain the same combination of variable - value mapping for certain
variables.

Due to the nature of Unison and the Gecode constraint solver calling the implementations
\textit{transformations}, which is the common nomenclature in software diversity, would be
misleading. No code is being explicitly transformed, combinations are instead implicitly
discarded when it is discovered that the constraints cannot be satisfied given a previous
branching.

\subsection{Strategies}
\label{sec:strategies}

For the purposes of this thesis we define three strategies to be evaluated and compared.
The goal of each strategy is to provide a population of as diverse versions of an
executable as possible while incurring as little overhead as possible. The chosen
strategies are \textit{enumeration}, \textit{instruction schedule} and
\textit{register allocation}, and the motivation for each is presented in each respective
subsection.

\subsubsection{Enumeration}

The name of this strategy comes from the fact that we are only concerned that the solutions
are different, not how they differ. During search the Gecode search engine will never
explore the same combination twice, and thus never generate two equal solutions. The
strategy is thus to not post any constraints at all and let the solver generate all
possible combinations. We \textit{enumerate} the solutions.

Unison can differ the solutions in four main ways:

\begin{itemize}
	\item The order of the operations is different
	\item Operands are connected to different registers
	\item Execute a copy using a different instruction (or not at all)
	\item Split live-ranges and spill temporaries differently
\end{itemize}

The results of this strategy serve as a baseline for the program. It is literally what
happens if we do nothing.

\subsubsection{Registers}

The strategy to diversify the register allocation of the resulting binaries is an attractive
one due to causing no run-time overhead. Consequently, if it introduces significant diversity
it is an excellent candidate. In addition, as register allocation is one of Unison's
primary purpose it feels like a natural strategy to explore.

There are two variables and one set that are of concern when diversifying the register
allocation. Their description from the Unison documentation are as follows:

\vspace{0.2cm}

\noindent\makebox[\textwidth]{
	\begin{tabular}{c|lr}
		\textbf{Type} & \textbf{Name} & \textbf{Description} \\ \hline
		\textbf{Set} & $P$ & The set of all operands in the program \\ \hline
		\multirow{2}{*}{\textbf{Variable}}
			& $x_p \in \{0, 1\}$ & whether operand \textit{p} is connected (0 is false and 1 is true) \\
			& $ry_p \in \mathbb{N}_0$ & register to which operand \textit{p} is assigned \\
	\end{tabular}
}

\vspace{0.2cm}

In pseudo mathematical notation we want to post the constraint:

\vspace{0.2cm}
\noindent\makebox[\textwidth]{
	$\neg (\bigwedge\limits_{p \in P} ( (x_p = prev.x_p) \land (ry_p = prev.ry_p) ))$
}
\vspace{0.2cm}

Very important to note is that the variables preceded by \textit{prev.} (i.e $prev.x_p$ and
$prev.ry_p$) represents an actual value. More precisely the value that is part of the previous
solution. For example $prev.xy_p$ represents the register to which operand $p$ is assigned
in the previous solution. The variables not preceded by \textit{prev.} are variables in
the constraint programming sense and their corresponding domains make up the remaining
combinations to explore.

In words; We disallow the exact same combination of connected (used) operands and
operand to register mapping.

The $ry_p$ variable of the model is actually an auxiliary variable that combines the
$r_t$ and $y_p$ variables. The $r_t$ variable represents which register temporary $t$ is
assigned, and $y_p$ represents which temporary is connected to operand $p$. In other words,
$ry_p$ is implemented as $r(y(p))$. This distinction is important during search, and in
particular when branching.

\subsubsection{Instruction Schedule}
\label{sec:schedule}

Given how Unison functions diversifying the instruction schedule is an exciting strategy.
As mentioned in Section \ref{sec:unison} Unison explores optional copies. In practice this
means that during pre-processing optional \textit{operations} are inserted so not only
does Unison decide on the order the instructions are executed, but in a limited capacity
Unison also inserts instructions (or deems instructions unnecessary). For the purposes
of breaking gadgets shifting instructions can help immensely as an adversary is reliant
on the exact addresses of the gadgets.

There are two variables and one set of interest for this strategy. In the Unison
documentation they are described as follow:

\vspace{0.2cm}

\noindent\makebox[\textwidth]{
	\begin{tabular}{c|lr}
		\textbf{Type} & \textbf{Name} & \textbf{Description} \\ \hline
		\textbf{Set} & $O$ & The set of all operations in the program \\ \hline
		\multirow{2}{*}{\textbf{Variable}}
			& $a_o \in \{0, 1\}$ & whether operation \textit{o} is active (0 is false and 1 is true) \\
			& $c_o \in \mathbb{N}_0$ & issue cycle of operation \textit{o} \\
	\end{tabular}
}

\vspace{0.2cm}

In order to disallow the same set of active operations combined with the same issue cycles
we post the constraint: 

\vspace{0.2cm}
\noindent\makebox[\textwidth]{
	$\neg ( (\bigwedge\limits_{o \in O} (a_o = prev.a_o)) \land (\bigwedge\limits_{m \in \{O | prev.a_m = 1\}} (c_m = prev.c_m)) )$
}
\vspace{0.2cm}

Just as for the registers strategy constraint, the variables preceded by \textit{prev.}
represents the actual value that is part of the previous solution, whereas the ones not
preceded by \textit{prev.} are the constraint programming variables used when searching for
future solutions.

The constraint described with words is that we disallow the exact same combination of
active operations and their corresponding issue cycle. We are only concerned about the
issue cycle of the previous solution's active operations, but we do take into account that
future solutions can have the same instructions issued at the same cycles if it has fewer
or more active operations.

As mentioned in Section \ref{sec:unison}, some operations can be implemented by multiple
instructions. For the purpose of breaking gadgets we do not want to allow functionally
equivalent instructions. While they would indeed make the sequence of bytes differ, the
functionality would be the same and the gadget would survive.


\section{Solution Distance}
When exploring combinations with a constraint solver similar solutions are found close to
each-other (in-time).  The number of possible combinations for certain strategies might be
far too large to handle efficiently so when limiting the number of binaries generated it is
beneficial to discard an amount of binaries in between those kept. The contrasting approach
is of course to stop the search after X amount of solutions are found or Y amount of time
has passed, in which case the solutions might be too similar.

\section{Performance}

Unison's main purpose is to generate the most optimal solution. In an effort to achieve
this Unison accepts the basic LLVM-solution as an optional parameter, and posts constraints
to only generate solutions that are \textit{better}. Better in this case is either that
they can be executed in fewer cycles or the size of the binary is smaller. Which is optimized
for is specified as a parameter.

For our purposes, we could limit our diversification strategies to only generate solutions
that execute in fewer (or the same amount of) cycles than the LLVM solution. In other words
we could generate executables with zero overhead \textit{with respect to LLVM's solution}.
Certain strategies would of course have an overhead, but with respect to the \textit{optimal
solution}.

\section{Architecture}

Unison does not currently support the x86 or x86-64 architecture. Only ARM, Hexagon and MIPS
are supported \cite{unison-src}. None of the supported architectures are complex instruction
set architecutres and thus hidden gadgets are not a problem. However, regular gadgets are
still present and for the purposes of systematic strategies for diversity the supported
architectures are sufficient.

\section{Evaluation}

Three key metrics will be evaluated. \textit{Diversity space}, \textit{estimated runtime},
and \textit{surviving gadgets}.

Diversity space and estimated runtime are simple metrics. The diversity space is simply
the total number of solutions found, and the estimated runtime is the issue cycle of the
last scheduled instruction.

Surviving gadgets is chosen as a measure for diversity. This is a natural metric given 
that the schedule strategy is chosen specifically due to being comparable to the work done
by \textcite{large-scale-automated} and that is the measurement they chose. Their test
was constructed as follows:

\begin{itemize}
	\item Generate an executable
	\item Apply transformations to end up with 30 different versions
	\item Compare the transformed executable to the original and count surviving gadgets
		\begin{itemize}
			\item A surviving gadget is a functionally equivalent sequence of instructions of
			length at the same offset.
		\end{itemize}
\end{itemize}

For our purposes the process is modified such that instead of generating one executable
and applying transformation multiple versions will be generated directly and then tested
pairwise.

The program to be evaluated is 433.milc from the SPEC2006 benchmark, as \textcite{large-scale-automated}
found it to be representative of the whole suite. It is fairly large (234 functions) and
thus for feasibility I will evaluate X versions at solution distance Y.
