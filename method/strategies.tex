\section{Strategies}

For the purposes of this paper only a few strategies will be evaluated. The chosen
strategies are \textit{difference}, \textit{instruction schedule} and
\textit{register allocation}, and the motivation for each is presented in respective
subsection.

As described in \ref{sec:unison} the problem of integrated register allocation and
instruction scheduling is modeled around \textit{operations} and \textit{operands}. The
problem consists of around 20 variables (see \ref{sec:constraint}) in total so only the
ones constrained will be presented for each strategy. These variables describe how the
operations and their operands relate to the actual instructions, temporaries, registers
and issue cycles.

\subsection{Evaluation}

Surviving gadgets,
Static analysis of cycles (or even actual runtime?),
Diversity space


\subsection{Gecode Constraint Solver}


Gecode is implemented such that a problem is modeled in a class inheriting from the
\textit{Space} base class. It is in this derived class variables are defined and constraints
are posted. When a solution is found by the Gecode branch and bound engine the virtual
member function \textit{constrain(b)} is called, where \textit{b} is the latest solution.
This function is expected to post constraints on future solutions based on the latest
solution. These constraints accumulate so posting constraints based on the previous solution
is sufficient for diversity.

Each strategy is implemented by constraints that, when satisified, ensures the
generated solutions differ in the correct manner. These constraints are posted in the
\textit{constrain()} function. In general the constraints will make sure that future
solutions does not contain the same combination of variable - value mapping for certain
variables.

Due to the nature of Unison and the Gecode constraint solver calling the implementations
\textit{transformations}, which is the common nomenclature in software diversity, would be
misleading. No code is being explicitly transformed, combinations are instead implicitly
discarded when it is discovered that the constraints cannot be satisfied given a previous
branching.


\subsection{Difference}

The name of this strategy comes from the fact that we are only concerned that the solutions
are different, not how they differ. During search the Gecode search engine will never
explore the same combination twice, and thus never generate two equal solutions. The
strategy is thus to not post any constraints at all and let the solver generate all
possible combinations.

Unison can differ the solutions in four main ways:

\begin{itemize}
	\item The order of the operations is different
	\item Operands are connected to different registers
	\item Execute a copy using a different instruction (or not at all)
	\item Split live-ranges and spill temporaries differently
\end{itemize}

The results of this strategy serve as a baseline for the program. It shows us how many
executables Unison can, at most, generate for the program. In other words, the biggest
diversity space for a given program (when using Unison).

\subsection{Instruction Schedule}

Given the work of \textcite{large-scale-automated} we are given an opportunity to compare
systematic diversity to randomized diversity. Generating all possible schedules is directly
comparable to randomizing the instruction schedule. It is interesting to see how well
the systematic approach performs, and how it must be tweaked to achieve results similar
to the randomizing approach.

There are three variables of interest for this strategy. In the Unison documentation
they are described as follow:

\vspace{0.2cm}

\begin{tabular}{cc}
	\textbf{Name} & \textbf{Description} \\
	$a_o \in \{0, 1\}$ & whether operation \textit{o} is active \\
	$i_o \in instrs(o)$\footnote{instrs(o) is the set of instructions that can implement operation o} & instruction that implements operation \textit{o}  \\
	$c_o \in \mathbb{N}_0$ & issue cycle of operation \textit{o} \\
\end{tabular}

\vspace{0.2cm}

Also required is the set $O$, which consists of all operations in the program.

\vspace{0.2cm}

There are two aspects to diversifying the instruction schedule; Either the sequence of
operations is different, or if they are equal at least one operation is implemented by a
different instruction.

In psuedo mathematical notation we want to post the constraint:

\vspace{0.2cm}

$\neg( (a_o = b.a_o) \land (i_m = b.i_m) \land (c_m = b.c_m) ) \quad \forall o \in O, \forall m \in \{ O | a_m = 1 \}$

\vspace{0.2cm}

Very important to note, and why I call it psuedo mathematical notation, is that the variables
preceded by \textit{b.} (i.e $b.a_o, b.i_m and b.c_m$) represents an actual value. More
precisely the value that is part of the previous solution. For example $b.c_m$ represents
the issue cycle of operation m in the previous solution. The variables not preceded by
\textit{b.} are variables in the constraint programming sense and their corresponding domains
are make up the remaining combinations to explore.

The constraint described with words is that we disallow the exact same combination of
active operations, issue cycles and instructions that implement the operations. We are
only concerned about the issue cycles and implementing instruction of active operations,
but we do take into account that future solutions can have the same instructions issued
at the same cycles if it has fewer or more active operations.


\subsection{Registers}

Constraining each binary to allocate registers in a different way is, if sufficient, an
attractive strategy due to causing no overhead. The downside for a production release is
minimal since there are usually no use-cases where knowing the register allocation is required.

In addition to creating the abstract \textit{operation} construct for an instruction, a
similar construct is also created for the temporaries, aptly named an \textit{operand}.
An operand can be connected to different temporaries between different solutions. In other
words, to ensure different register allocation between solutions we post constraint that,
when satisfied, ensure that not all \textit{operands} are connected to the same temporaries,
and if they are, at least one temporary is allocated to a different register.
