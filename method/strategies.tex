\section{Strategies}

For the purposes of this paper only a few strategies will be evaluated. When exploring
combinations with a constraint solver similar solutions are found close to each-other (in-time).
The number of possible combinations for certain strategies might be far too large to handle
efficiently so when limiting the number of binaries generated it is beneficial to discard
an amount of binaries in between those kept. The contrasting approach is of course to stop
the search after X amount of solutions are found or Y amount of time has passed, in which
case the solutions might be too similar.


\subsection{Constraining Difference}

The trivial strategy for diversification with Unison is to simply require all executables
to be \textit{different}. Unison can differ code in a few ways (see \ref{sec:unison}):

\begin{itemize}
	\item The order of the instructions is different
	\item Temporaries are connected to different registers
	\item Execute a copy using a different instruction (or not at all)
	\item Split live-ranges and spill temporaries differently
\end{itemize}

The triviality of this strategy comes from the ease of implementation. We simply don't post
any more constraints when branching, which will cause all valid solutions to be generated.

This strategy is expected to have \textit{a lot} of solutions. Several millions of them
for a large input.

\subsection{Instruction Schedule}

For reference to \textcite{large-scale-automated} diversifying the instruction schedule will
be implemented. This would be the equivalent of randomizing the instruction schedule. It is
interesting to see the size of the diversity space (i.e number of binaries that can be
generated) and how different the schedule must be to achieve similar results.

Unison operates by creating an abstract construct, an \textit{operation}, for every instruction
in the MIR input. One reason for this is that some instructions can be replaced with
another instruction and be functionally equivalent, but (hopefully) faster. This is also
the way Unison implements alternative moves (and allows for live-range splitting).
Operations can be flagged as not-used, so in the optimal solution nothing unnecessary will
be executed. In other words some operations can be implemented by several instructions.
When implementing the diversification of the instruction schedule this is taken into
account and the constraints posted when a solution is found entails making sure that
future solutions do not have the same sequence of operations, and if they do that at least
one operation is impemented by another instruction.

\subsection{Registers}

Constraining each binary to allocate registers in a different way is, if sufficient, an
attractive strategy due to causing no overhead. The downside for a production release is
minimal since there are usually no use-cases where knowing the register allocation is required.

Unison keeps track of which operand is allocated to which register. To ensure different
register allocations we post constraints that, when satisfied, ensures the registers assigned
to each operand are different for the same active operands.
