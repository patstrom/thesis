\label{sec:strategies}

For the purposes of this thesis we define three strategies to be evaluated and compared.
The goal of each strategy is to provide a population of as diverse versions of an
executable as possible while incurring as little overhead as possible. The chosen
strategies are \textit{enumeration}, \textit{instruction schedule} and
\textit{register allocation}, and the motivation for each is presented in each respective
subsection.

\subsection{Enumeration}

The name of this strategy comes from the fact that we are only concerned that the solutions
are different, not how they differ. During search the Gecode search engine will never
explore the same combination twice, and thus never generate two equal solutions. The
strategy is thus to not post any constraints at all and let the solver generate all
possible combinations. We \textit{enumerate} the solutions.

Unison can differ the solutions in four main ways:

\begin{itemize}
	\item The order of the operations is different
	\item Operands are connected to different registers
	\item Execute a copy using a different instruction (or not at all)
	\item Split live-ranges and spill temporaries differently
\end{itemize}

The results of this strategy serve as a baseline for the program. It is literally what
happens if we do nothing.

\subsection{Registers}

The strategy to diversify the register allocation of the resulting binaries is an attractive
one due to causing no run-time overhead. Consequently, if it introduces significant diversity
it is an excellent candidate. In addition, as register allocation is one of Unison's
primary purpose it feels like a natural strategy to explore.

There are two variables and one set that are of concern when diversifying the register
allocation. Their description from the Unison documentation are as follows:

\vspace{0.2cm}

\noindent\makebox[\textwidth]{
	\begin{tabular}{c|lr}
		\textbf{Type} & \textbf{Name} & \textbf{Description} \\ \hline
		\textbf{Set} & $P$ & The set of all operands in the program \\ \hline
		\multirow{2}{*}{\textbf{Variable}}
			& $x_p \in \{0, 1\}$ & if operand \textit{p} is connected (false/true) \\
			& $ry_p \in \mathbb{N}_0$ & register to which operand \textit{p} is assigned \\
	\end{tabular}
}

\vspace{0.2cm}

In pseudo mathematical notation we want to post the constraint:

\vspace{0.2cm}
\noindent\makebox[\textwidth]{
	$\neg (\bigwedge\limits_{p \in P} ( (x_p = prev.x_p) \land (ry_p = prev.ry_p) ))$
}
\vspace{0.2cm}

Very important to note is that the variables preceded by \textit{prev.} (i.e. $prev.x_p$ and
$prev.ry_p$) represents an actual value. More precisely the value that is part of the previous
solution. For example $prev.xy_p$ represents the register to which operand $p$ is assigned
in the previous solution. The variables not preceded by \textit{prev.} are variables in
the constraint programming sense and their corresponding domains make up the remaining
combinations to explore.

With other words we disallow the exact same combination of connected (used) operands and
operand to register mapping.

The $ry_p$ variable of the model is actually an auxiliary variable that combines the
$r_t$ and $y_p$ variables. The $r_t$ variable represents which register temporary $t$ is
assigned, and $y_p$ represents which temporary is connected to operand $p$. In other words,
$ry_p$ is implemented as $r(y(p))$. This distinction is important during search, and in
particular when branching.

\subsection{Instruction Schedule}
\label{sec:schedule}

Given how Unison functions diversifying the instruction schedule is an exciting strategy.
As mentioned in Section \ref{sec:unison} Unison explores optional copies. In practice this
means that during pre-processing optional \textit{operations} are inserted so not only
does Unison decide on the order the instructions are executed, but in a limited capacity
Unison also inserts instructions (or deems instructions unnecessary). For the purposes
of breaking gadgets shifting instructions can help immensely as an adversary is reliant
on the exact addresses of the gadgets.

There are two variables and one set of interest for this strategy. In the Unison
documentation they are described as follow:

\vspace{0.2cm}

\noindent\makebox[\textwidth]{
	\begin{tabular}{c|lr}
		\textbf{Type} & \textbf{Name} & \textbf{Description} \\ \hline
		\textbf{Set} & $O$ & The set of all operations in the program \\ \hline
		\multirow{2}{*}{\textbf{Variable}}
			& $a_o \in \{0, 1\}$ & if operation \textit{o} is active (false/true) \\
			& $c_o \in \mathbb{N}_0$ & issue cycle of operation \textit{o} \\
	\end{tabular}
}

\vspace{0.2cm}

In order to disallow the same set of active operations combined with the same issue cycles
we post the constraint: 

\vspace{0.2cm}
\noindent\makebox[\textwidth]{
	$\neg ( (\bigwedge\limits_{o \in O} (a_o = prev.a_o)) \land (\bigwedge\limits_{m \in \{O | prev.a_m = 1\}} (c_m = prev.c_m)) )$
}
\vspace{0.2cm}

Just as for the registers strategy constraint, the variables preceded by \textit{prev.}
represents the actual value that is part of the previous solution, whereas the ones not
preceded by \textit{prev.} are the constraint programming variables used when searching for
future solutions.

The constraint described with words is that we disallow the exact same combination of
active operations and their corresponding issue cycle. We are only concerned about the
issue cycle of the previous solution's active operations, but we do take into account that
future solutions can have the same instructions issued at the same cycles if it has fewer
or more active operations.

As mentioned in Section \ref{sec:unison}, some operations can be implemented by multiple
instructions. For the purpose of breaking gadgets we do not want to allow functionally
equivalent instructions. While they would indeed make the sequence of bytes differ, the
functionality would be the same and the gadget would survive.
