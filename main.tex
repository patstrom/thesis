\documentclass{kththesis}

\usepackage{csquotes} % Recommended by biblatex
\usepackage{biblatex}

\addbibresource{background/software-diversity/software_diversity.bib}
\addbibresource{background/llvm/llvm.bib}
\addbibresource{background/constraint-programming/constraint_programming.bib}
\addbibresource{background/unison/unison.bib}


\usepackage{graphicx}
\usepackage{listings}
\usepackage{xcolor}

\usepackage{dirtytalk}

\usepackage[hidelinks]{hyperref}

\usepackage{amsfonts}
\usepackage{multirow}
\usepackage{longtable}
\usepackage{textcomp}
\usepackage{seqsplit}



\title{A Systematic Approach to Automated Software Diversity using Unison}
\alttitle{Ett Systematiskt tillvägagångsätt för Automatiserad Mjukvarumångfald med Unison}
\author{Patrik Karlström}
\email{pkarlstr@kth.se}
\supervisor{Benoit Baudry}
\examiner{Christian Schulte}
\programme{Master in Embedded Software}
\school{School of Electrical Engineering and Computer Science}
\date{\today}


\begin{document}

% Frontmatter includes the titlepage, abstracts and table-of-contents
\frontmatter

\titlepage

% English abstract
\begin{abstract}
	English abstract goes here.
\end{abstract}



% Swedish abstract
\begin{otherlanguage}{swedish}
  \begin{abstract}
		Unison är ett verktyg som kombinerar instruktionsschemaläggning och registerallokering
		som ett enat kombinatoriskt problem och löser det med villkorsprogrammering, vilket
		är en programmeringsparadigm för att systematiskt lösa kombinatoriska problem.

		Automatiserad mjukvarumångfald innefattar att generera varierade exekverbara filer i
		ansats att bryta så kallade \textit{gadgets}, vilket är korta instruktionssekvenser
		som utgör en attackvektor. Attacker som utnyttjar gadgets förlitar sig mycket på
		filens interna struktur. Genom att generera en population av exekverbara filer med
		ekvivalent funktionalitet men strukturella skillnader så måste en motspelare
		konstruera en unik attack för varje exekverbara fil.
		
		Att använda Unison för att systematisk generera varierade exekverbara filer visar att
		antalet möjliga, parvis distinkta exekverbara filer är ofta större än 1000000, även
		för små funktioner (<10 instruktioner). Att använda Unison för att tvinga de
		exekverbara filerna att skilja sig på specifika sätt är enkelt att implementera, bara
		en handfull rader kod.  En utvärderad strategi i experimentet resulterade i att den
		gadget med högst frekvens återfanns i 24\% av alla programversioner, och 82\% av alla
		gadgets återfanns i endast en programversion var. Dock så krävs mer arbete innan någon
		konsumentorienterat kan utvärderas, delvis för att Unison inte stödjer
		processorarkitekturen x86.
	\end{abstract}
\end{otherlanguage}


\renewcommand{\abstractname}{Acknowledgements}
\begin{abstract}
I am very grateful to both Benoit Baudry and Christian Schulte at the SCS department of
KTH for their support, without which this project would not have been possible.

A special thanks is owed to Roberto Castañeda Lozano, researcher at the Computer Systems
Laboratory of SICS and the main author of Unison, who has provided invaluable assistance
with both devising and implementing my model. Thank you, Roberto!
\end{abstract}


\tableofcontents


% Mainmatter is where the actual contents of the thesis goes
\mainmatter

% Introduction
\chapter{Introduction}

The field of \textit{software diversity} is concerned with researching the
causes and effects of diversity in software and software engineering, both in the result
and the process. As an example consider web browsers. There are many different implementations
of what is essentially the same functionality. In the context of security vulnerabilities
of one browser does not necessarily carry over to another \cite{survey}.

One of the applications of software diversity is as a defense to code re-use attacks \cite{survey}.
Code-reuse attacks refers to attacks that diverts program control flow to re-use already
present code in an unintended manner \cite{code-re-use}. For example if an adversary gains
access to a process' stack the address to jump to after a return instruction can be
overwritten and the adversary can choose which instruction to jump to. \textcite{rop}
extended this concept and introduced what is called \textit{return-oriented programming}.
Return oriented programming is a code re-use technique were the program control flow is
diverted to a chain of short instruction sequences, dubbed a \textit{gadget}. By carefully
choosing these gadgets an adversary can achieve arbitrary code execution.

However, the adversary relies on the fact that the chosen instruction sequences are always
equivalent across all binary files. In other words an equivalent sequence of instructions
is located at the exact same offset. If, for example, everyone would run their own,
structurally different but functionally equivalent version of the Firefox web browser an
adversary would have to disassemble all variations and construct a unique payload for
every target. Thus, techniques that provide diversity between executables and/or runtimes
have been researched as a defense to these code re-use attacks \cite{survey}.

To understand how two executables can be structurally different but functionally equal
consider an example pertaining the instruction schedule. Say there are two independent
instructions, one of which is part of a gadget. By swapping the place of these two
instructions the gadget is now different, but since the instructions are independent
program functionality is the same. There are many decisions taken by the compiler that are
not necessarily the only valid decision, and if one could explore all combinations of
decisions one could systematically generate the entire population of diversely arranged
but functionally equivalent executables.

Constraint programming is a paradigm for solving combinatorial search problems wherein
the relationship between variables are expressed as \textit{constraints}. It solves problems
such as scheduling, vehicle routing and, in our case, compiling
 \cite{handbook-constraint-programming, unison-docs}. For example, consider the problem
of scheduling working shifts in a store. The variables could be the start and end time
of a shift for one person, and the constraint could be that there are always two employees
present in the store and no one works for more than 8 consecutive hours.

A \textit{constraint solver} takes a problem such as the example above and finds a value
for all the variables such that all constraints are satisfied, or in the case when no
mapping that satisfies all constraints exists proves that such is the case
 \cite{handbook-constraint-programming}. The process of finding these values consists of
\textit{searching} the combinatorial space and either implicitly or explicitly discarding
combinations that cannot be solutions, i.e. does not satisfy the constraints.

\textit{Unison} is a tool that uses a constraint solver to schedule instructions and allocate
registers as a single, integrated problem. It operates as a part of the LLVM backend, and
is thus not a standalone compiler. Unison is potentially optimal in the sense that given
enough time, it can generate the optimal instruction schedule and register allocation.
It finds this solution by searching the combinatorial space in a continuous pursuit for a
solution that is better than the last \cite{unison-docs}.

\section{Problem Statement}

It might feel intuitive to provide software diversity by somehow randomizing a part of an
executable, and in fact many have \cite{survey,librando,binary-stirring}, but this
approach comes with a three key challenges: It is difficult to make guarantees about the
resulting population, proving the correctness of the code transformations is challenging
and there is no fine-grained control of the process.

When generating the executables with a random element in the process you cannot make any
guarantees regarding the resulting population. If you generate two executable you cannot
with certainty tell that they are in fact sufficiently different without verifying it. The
risk of them being equal is most likely very small for just two versions, but what if
10000 different version are generated? Or 1000000? At some point the pigeonhole principle
comes into effect, in which case case it would be better to re-use the already generated
code.

By instead systematically generating different variations those sort of guarantees can be
made. If two equal executables are never generated it follows that they are all pairwise
distinct. It also follows that if the code generator is run to termination, all pairwise
distinct versions are generated. There is also opportunity to define exactly what pairwise
distinct entails. The nature of constraint solving lends itself well to this application.
This is where Unison comes in. In addition to its main purpose Unison has potential for
software diversity purposes.

This thesis aims to be an early exploration of a systematic approach to automated software
diversity using a constraint solver, specifically the Unison tool. It aims to evaluate
both the code generator and the generated code. This leads us to the hypothesis that

\say{Systematically generating diverse executables using a constraint solver is at least
as good as basing the diversification process on randomization.}

\section{Motivation}

As mentioned, it is difficult to prove the correctness of code transformations. Perhaps
the most compelling reason to use a constraint solver for software diversity is that for
a combination to be considered a solution, \textit{all} constraints must be satisfied.
Regardless of what unorthodox approach we want to try out we do not need to consider it's
effect on program functionality, since the constraints that ensure a valid program is
generated must still be satisfied. This property makes implementing new techniques
relatively simple. Modifying code while retaining functionality is no easy task, and this
simplifies it greatly.

With the systematic approach there is full control of what changes between versions, and
how much it changes between versions. Instead of introducing randomness into some part of
the compilation process and more or less hope it generates diverse code with low overhead
the systematic approach offers the possibility of properly reasoning about the process and
steering it specific manners. It also yields a greater control of the quality of the
generated code. Limits can be dynamically set in place so that it does not exceed e.g a
certain number of instructions.

\section{Methodology}

This project is an early exploration of the potential benefits of using a constraint solver
for automated software diversity. Specifically, the research will be carried out by
modifying the Unison tool to support generating multiple outputs. Exactly how these
outputs should differ to generate the most diverse population is outside the scope of this
project. With that said, for the purposes of this thesis the modification of Unison will
include three different options of how the resulting code should differ. Both the modified
code generator as well as the resulting population of each of the three options will be
evaluated. This lends itself well to an experimental approach.

The purpose of the experiment is to evaluate both the code generator and the resulting
code. The evaluation of the code generator will be performed on both quantifiable and
non-quantifiable metrics (such as ease of implementation). The resulting code, which will
be a population of functionally equivalent but differently arranged code, will be evaluated
in a purely quantifiable way.

\section{Ethics and Sustainability}

Developing defenses in the field of software security does not generally offer many
ethical controversies. It is in principle always good to offer defense against ominous
deeds. However, generating multiple executables on each invocation of a compiler would
lead to sustainability impacts.

\textcite{compiler-generated-sw-div} describes the architecture for a solution like this
when deployed on a large scale. It involves an "app store" (e.g a package manager
repository or a smartphone app store) where a user asks for a binary and each user is given
a unique variant. The two options are to either generate a new version when one is asked
for or to generate as many versions as feasible beforehand, storing them somewhere and
simply providing one at user requests. Both of these approaches involves more computation
than just compiling a single binary and providing it to all users. In the latter case a
lot of storage is also involved.

Initially either of these factor might not seems like much but requiring additional
computation and potentially storage that scales linearly with users can become a problem
for popular applications, such as certain web browsers or common software libraries. The
extra computation and storage requires power which is costly not only in monetary terms
but also environmentally. Unfortunately, this is an unavoidable consequence of a large
scale deployment of automated software diversity.

\section{Overview}

This thesis is divided into five chapters. The first chapter provides an overview and
introduction of the project background and goal as well as the research method and ethics
and sustainability issues.

Chapter 2, background, presents software diversity and why it is desirable. Also presented
is how the systematic approach will be taken in the form of Unison and constraint
programming.

Chapter 3 and 4 of the thesis consists of, respectively, what experiment has
been done and the results of that experiment. The last chapter discusses the shortcomings
of the experiment, the future work required and the conclusion.


% Background
\chapter{Background}

This section will be about contraint programming, software diversity, LLVM and Unison

% Software diversity section
\section{Software Diversity}

Software diversity is a diverse field, and there is research focusing on different areas
with different goals in mind. However, what they all have in common is the that they are
exploring the potential benefits of engineering diversity.

% https://softwarediversity.eu/survey.pdf

\subsection{Goals of Software Diversity}

% Why do we want this crap anyway?

\subsection{Managed Software diversity}

% Design diversity (N-version) https://softwarediversity.eu/survey.pdf 3.1
% Natural. Mention stuff like the browser and perhaps different DBMS
% Functional. Abstract over implementation details (e.g OS)

% Don't do a lot of stuff related to this. Less emphasis here!

\subsection{Automated Software Diversity}

% This is where we are!
% More emphasis here!

% https://wkr.io/public/ref/wartell2012stirring.pdf
% https://www.informatik.tu-darmstadt.de/fileadmin/user_upload/Group_TRUST/PubsPDF/readactor.pdf


% Contraint programming section
\section{Constraint Programming}

Constraint programming is a programming paradigm for solving combinatorial problems.
By declaring all variable's possible values and their \textit{constraints}, the relationship
between them when part of a solution, a \textit{constraint solver} can search the solution
space and find an assignment of variables to values that is consistent with the constraints.
A constraint solver effectively explores different possible combinations systematically,
by a potentially incomplete local inference (also known as \textit{constraint propagation})
or more commonly a combination of the two \cite{handbook-constraint-programming}.

Currently the only constraint solver supported by Unison (which is the primary tool used
in this paper. See \ref{sec:unison}) is Gecode\footnote{\url{www.gecode.org}}
\cite{unison-docs}, which is a constraint solver that interleaves systematic search algorithms
with constraint propagation\cite{MPG}.

When propagating constraints the Gecode solver searches all variable's domains and removes
variables that are in conflict with the constraints\cite[Section~23.1]{MPG}. For example,
given two variables (and their corresponding domains) $x \in {0,1,2}$ and $y \in {0,1,2}$
and the constraint $x > y$ constraint propagation can determine that $x \in {1, 2}$ and
$y \in {0, 1}$ are the only combinations consistent with the constraint.

When constraint propagation is finished there are three possible states:

\begin{enumerate}
	\item One or more domains could be empty, proving that no solution exists.
	\item	All domains could be of size 1, indicating that there exists only one possible
		value for every variable, and thus we have found a solution.
	\item One or more variables have multiple values in their domain.
\end{enumerate}

For situation (1) and (2), either a solution is found or we have proven that a solution
does not exist (within the local search space).

In the latter situation (3) the Gecode constraint solver splits a variable's domain into
two or more subsets, creating a \textit{search tree} where each \textit{branch} represents
reducing the variable's domain to a particular subset \cite[Section~8]{MPG}. By committing
to a branch the constraint solver can once again perform constraint propagation and repeat
the process. However, if \textit{branching} has taken place and the solver reaches
situation (1) it can go back up the tree and explore a different branch. If situation (2)
is reached it can still backtrack but with the added option of adding more constraints
based on the newly found solution. Constraining based on previously found solutions in the
search tree is done with the \textit{branch-and-bound} search engine\cite[Section~9]{MPG}
in Gecode.

\textit{Branch and bound} is an efficient strategy to find the optimal solution to a
combinatorial problem which in essence constitutes comparing potential solutions to the
currently best found solution, choosing the better of the two\cite{BaB}. We will adopt a
similar strategy, but instead of constraining solutions to be better we will require them
to be \textit{different}. In addition, no solutions will be discarded.


% LLVM
\section{LLVM}
LLVM is an umbrella project that provides a collection of tools for developing low-level
toolchains, e.g assemblers, compilers, debuggers, linkers etc. It is designed to be reusable and
applicable to arbitrary programming languages and target architectures. It started as a
research project at the University of Illinois in 2000 and is widely used today by hobbyists
and professionals alike. There are LLVM frontends for languages such as Haskell, Rust,
Swift and Ruby. Clang is also a part of the LLVM project and built upon
the LLVM toolchain. It provides a compiler, debugger and standard library implementation
for the C language family (C, C++, Objective-C, OpenCL, Cuda etc).

% http://www.rubymotion.com/tour/features/
% https://ghc.haskell.org/trac/ghc/wiki/Commentary/Compiler/Backends/LLVM
% http://llvm.org/ https://en.wikipedia.org/wiki/Clang


\subsection{Overview of LLVM}

% http://www.aosabook.org/en/llvm.html
LLVM is designed to be modular. A compiler written on top of LLVM will in general consist
of three phases; The front-end, the mid-end (also called optimizer), and the back-end
(also known as code generator). The front-end is responsible for lexical and syntatical
analysis of the source code. The mid-end is responsible for target-indenpendent optimizations
and the back-end handles platform specific tasks such as instruction selection, register
allocation and instruction scheduling. The point of LLVM is that the interface between
these modules are well-defined and thus allow for reuse so that, e.g, a front-end for
C uses the same optimizer as a front-end for Rust. In a similar manner the back-end
for x86 and the back-end for ARM uses the same mid-end. See figure \ref{fig:three_phase_compiler}.
LLVM can thus be seen as a \textit{collection of libraries} that perform, or at least
assists in performing, these different tasks.

\begin{figure}[h]
	\centering
	\includegraphics[width=12cm]{background/llvm/figures/three_phase_compiler}
	\caption{Three-phase compiler construction. The mid-end is somtimes called an \textit{optimizer}
	and the back-end a \textit{code generator}}
	\label{fig:three_phase_compiler}
\end{figure}

In the case of LLVM, the interface between the front-end and the mid-end is known as the
LLVM IR (\textit{Intermediate Representation}) and is a strongly typed, RISC-like virtual
instruction set that abstracts some details about the machine, such as function call
conventions and registers. To implement a programming language one would then only have to
implement a front-end, that is translate the source code to LLVM IR. One could then pick
and choose from the already existing LLVM optimizer passes and LLVM back-end implementations
to complete the compiler.

Only the back-end is of concern to this thesis, so it will focus exclusively on that.

\subsection{LLVM back-end}

A LLVM back-end, also known as a LLVM code generator, has a well defined input. The
LLVM IR. The output is often an object file, but a backend can also act as an interpreter.
For most of it's task, a LLVM backend uses what is called the \textit{machine IR} to represent
the code during the different stages.

% https://llvm.org/docs/WritingAnLLVMBackend.html
The LLVM back-end is responsible for three main tasks: intruction selection, instruction
scheduling and register allocation. They are performed mostly in isolation to each
other, which simplifies the architecture but introduces some interesting challenges.

The general behaviour of an LLVM back-end is that it transforms LLVM IR to machine code
for a specific target platform. Exactly what it produces depends on the target architecture
but it is common to generate assembly corresponding to the instruction set of the target
machine.

This paper does not cover instruction selection or other, smaller, back-end tasks such as
optimization, ABI implementation, exception handling etc.

\subsubsection{LLVM Machine Representation}

To represent machine specific IR LLVM uses what is called "Machine specific representation".
The machine specific representation consists of target agnostic, in-memory representations
of functions (MachineFuntion), basic blocks (MachineBasicBlock) and instructions (MachinrInstr).
% https://releases.llvm.org/3.8.1/docs/MIRLangRef.html#id8
% https://releases.llvm.org/3.8.1/docs/CodeGenerator.html#machine-code-representation

The machine specific representation can be represented as the LLVM MIR (\textit{Machine
Intermediate Representation}), which is a human readable, YAML serialized format. It is
used to test code generation passes in LLVM. That is, we can stop the LLVM back-end
prematurely and view the current progress in the form of an LLVM MIR.
% https://llvm.org/docs/MIRLangRef.html

For example, the iterative factorial function written in C:

\lstinputlisting[language=C,tabsize=2,frame=single,breaklines=true,showstringspaces=false,
backgroundcolor= \color{lightgray}]
{background/llvm/examples/factorial.c}

gets translated to the following MIR when targeting the Hexagon V4 architecture and
terminating after instruction selection, right before register allocation and instruction
scheduling.

\lstinputlisting[tabsize=2,frame=single,breaklines=true,showstringspaces=false,
backgroundcolor= \color{lightgray}]
{background/llvm/examples/factorial.mir}

Courtesy of the Unison documentation for the code examples.
% https://github.com/unison-code/unison/tree/master/doc/code

What is interesting to note is that there is an infinite number of virtual registers available,
starting from \%0 and incrementing, as well as some abstract intructions, such as the PHI
function.

\subsubsection{Instruction Scheduling}
% https://llvm.org/docs/CodeGenerator.html#scheduling-and-formation
% https://llvm.org/devmtg/2016-09/slides/Absar-SchedulingInOrder.pdf - find better souce than this. Needs to be abou different archiectures
Instruction scheduling is the task of choosing how and in which order the selected instructions
will run. In the case of LLVM the scheduling phase is logically seperate from the selection
phase. There exists a wide variety of architecture and they all have different limitations
and constraints.

Perhaps the most intuitive kind of architecture is the in-order processor where all
instructions are executed sequentially and the schedule is staticically generated.

Alternatives to this architecture is and out-of-order processor where all instructions are
fetched and committed in-order, but executed out-of-order. In this case the instructions
are dynamically scheduled by the CPU. You can also have a VLIW (\textit{Very Long Instruction
Word}) processor where multiple instructions can be statically scheduled in parallel.

% http://infolab.stanford.edu/~ullman/dragon/w06/lectures/inst-sched.pdf - find better source for stuff like architecture, instruction scheduling goals and stuff.

As previously mentioned LLVM is a set of libraries, and thus no one way to schedule instructions
in LLVM exists. However, the key factors to take into account during instruction scheduling
is generally to make use of architecture specific features (such as VLIW), avoiding pipeline
stalls by rearranging instructions, reducing register pressure etc. When scheduling instructions
the main constraint is data-dependencies. Some instructions depends on the result of previous
instructions, and must thus be scheduled sequentially. In the cases where these dependencies
does not exists (or can be broken by e.g introducing new temporaries) the instructions
can be scheduled in a different order or in parallel (if the hardware allows it).

Instruction scheduling can be seen as a problem where the input is a list of resources
(e.g ALUs, FPUs, presence of VLIW) and a execution cycles for each instruction, and the
output is the instruction in the sequence they should be executed for optimal performance
(with regards to whatever factor you are optimizing for). In other words, maximize \textit{
	instruction level parallellism}. Of course this problem is very difficult.

\subsubsection{Register Allocation}
% https://llvm.org/pubs/2009-04-SCOPES-RegisterAllocationDeconstructed.pdf
% https://llvm.org/docs/CodeGenerator.html#register-allocator

There are a few different problem concering register allocation. Namely, coalescing, spilling
move insertion and assigment.

Coalescing is the attempt to eliminate move instructions that refers to identical locations,
i.e. remove unnecessary moves. When spilling a register you allocate the variable on the
stack to free up the register for another temporary ("spilling" the temporary to the stack)
and move insertion is used to split the lifetime of temporaries.

In general the task is modeled as a graph-coloring problem. The first step is then to
perform a \textit{liveliness analysis}, where you analyse the temporaries in the program
to determine which values has to co-exist (i.e. be placed in different registers). With
the liveliness analysis in hand a graph can be constructed where each node is a temporary,
edges represents that the two temporaries must be assigned to distinct registers and the
colors are registers. E.g for a machine with 64 registers you would use at most 64 colors
to color the graph. Assigning a temporary to a specific register (perhaps because of
calling conventions) would then be to "pre-color" the corresponding node. Coalescing in
this model would be to "merge" two nodes and spilling would be to remove a node. Spilling
is usually a consequence of not being able to find a coloring of the graph, thus
necessitating that a node be removed.

\subsubsection{Problems with LLVM Approach}
% http://i.stanford.edu/pub/cstr/reports/cs/tn/95/22/CS-TN-95-22.pdf 
It is a common approach to treat instruction scheduling and register allocation as two
distinct steps in the back-end pipeline, and indeed so does LLVM. There are two approaches
to this, either you start with one or the other. Starting with instruction scheduling is,
nowadays, more common. Doing the register allocation first was mainly used in the early
days of compilers when machines didn't have as many registers.

% citation needed
% maybe here: https://pdfs.semanticscholar.org/5246/7f742a206b075324ec17b9b7f9d539f52ec8.pdf
Seperating these steps in such a manner leads to some problems, and usually some instruction
scheduling is once again performed after the register allocation. If a register needs to be
spilled then the dependency graph which we based our schedule on is no longer valid. Re-doing
these steps is quite expensive though, as you would essentially throw out all previous
progress as soon as a register is spilled. Instead, modern compiler generate suboptimal
code. Not very modern.

% Mention the problem with offset and having to scavenge register
% https://youtu.be/objxlZg01D0?t=48m53s
More subtle, machine specific problems can also occur even after both instruction scheduling
and register allocation has happened. When setting up stack frames and resolving frame
indices one might find oneself in the situation of needing both additional instructions
\textit{and} more registers. This can occur because of the targets limited number of immediates,
for examples we might be doing something like this:

\lstinputlisting[tabsize=2,frame=single,breaklines=true,showstringspaces=false,
backgroundcolor= \color{lightgray}]
{background/llvm/examples/initial_inst.mir}

Where we attempt to store a value given a (symbolic) frame index. This frame index will be replaced by
the LLVM Prolog Epilog Inserter to a stack pointer and an offset. For example it might
generate something like this:

\lstinputlisting[tabsize=2,frame=single,breaklines=true,showstringspaces=false,
backgroundcolor= \color{lightgray}]
{background/llvm/examples/stack_offset.mir}

where it has replaced the frame index with an offset of 4104 from the stack pointer. However,
if the target cannot encode 4104 directly as an immediate we need another instruction to
calculate this offset, and another register to store the offset in, like so:

\lstinputlisting[tabsize=2,frame=single,breaklines=true,showstringspaces=false,
backgroundcolor= \color{lightgray}]
{background/llvm/examples/temporary_resources.mir}

Thus, we are in a situation where we need more instructions and more register after both
instruction scheduling and register allocation is already over. LLVM solves this with something
they call \textit{register scavenging}, which is not something that will be covered here.

While this specific example might have needed to incorporate instruction selection into
the mix as well, it is nonetheless a difficulty.

% Citations for this. Unison, http://i.stanford.edu/pub/cstr/reports/cs/tn/95/22/CS-TN-95-22.pdf
% and find more.
Instruction selection, instruction scheduling and register allocation are all tasks that
depend on each other, but in order to make them managable by current solution they need
to be performed mostly in isolation to each other. Combining instruction scheduling and
register allocation is a popular subject of research.


% Unison
\section{Unison}

\subsection{The high-level approach of Unison}

How it models the problem

\subsection{Unison IR}

How it represents the problem

\subsection{What Unison does and doesn't do}

What we can do with it for our purposes



% Method
\chapter{Diversification with Unison}

In this chapter the relevant parts of the Unison model, the modification to the models that
will be implemented and the experimental protocol will be presented.

The first section, \textit{Automatic Diversity Synthesis with Unison} briefly presents the
Unison model and how to implement our extension of it. The \textit{disUnison} section
describes how we extend the model to provide software diversity, and the
\textit{Experimental Protocol} section describes how the experiment will be conducted.

\section{Automatic Diversity Synthesis with Unison}
\label{sec:unison-model}

In its current state Unison accepts input in the form of the LLVM MIR (see Section
\ref{sec:unison}) of a single function \cite{unison-docs} and outputs LLVM MIR of the same
function with allocated registers and scheduled instructions. It is then the job of
\textit{llc}, the LLVM static compiler to emit architecture specific assembly, which can
be passed through a native assembler and linker to generate an executable \cite{llvm-llc}.

As described in Section \ref{sec:unison} the problem of integrated register allocation and
instruction scheduling in Unison is modeled around \textit{operations} and \textit{operands}.
The problem consists of around 20 variables (see \ref{sec:constraint}) in total so only the
ones relevant for the experiment will be presented. These variables describe how the
operations and their operands relate to the actual instructions, temporaries, registers
and issue cycles.

The only constraint solver currently supported by Unison is Gecode \cite{unison-docs}. A
branch and bound search can be implemented in Gecode by using a branch and bound search
engine and overriding the virtual member function \textit{constrain()} of the model. The
\textit{constrain()} function is invoked by the search engine whenever a solution is found
and takes the most recently found solution as an argument. The idea is to post constraints
on the next solution based on the previous solution. These constraints accumulate so
that all future solutions will be affected by all already found solutions.

As mentioned, our intention is to replace the current \textit{constrain()} function of
Unison with one that instead posts constraints to ensure that future solutions are
\textit{different} in some regard. Remember that in the context of Unison a solution is a
valid instruction schedule and allocation of registers.

\section{Cost}
\label{sec:cost}

An important variable in the Unison model is \textit{cost}. It is the deciding factor
of whether a solution is better than another during the branch and bound search. It is a
sum of the estimated cost of each basic block, weighted by the estimated execution
frequencies \cite{unison-docs}. Cost can be either cycles or code size depending on if the
optimization goal is speed or size, respectively. That is the \textit{constrain()} function
of the Unison model post the constraint that for future combinations to be considered
solutions, in addition to the original constraints, the cost must be less than the cost of
the previously found solution. Given enough execution time, Unison will thus find the
optimal solution with regards to the \textit{cost} variable.

\section{disUnison}
\label{sec:disUnison}

The implementation that makes up the following strategies that are used in the experiment
is henceforth referred to as disUnison (from the word disunity). It is a variation of the
original Unison model, where the key difference is that the behaviour of the branch and
bound search is modified. The bulk of the Unison model, that ensures functional code is
emitted, is still present in the disUnison model.

\section{Strategies}
\label{sec:strategies}

% Revise: "In this thesis we define and compare three strategies to automatically
% diverse versions of the binary code for a given program
For the purposes of this paper only a few strategies will be evaluated. The chosen
strategies are \textit{enumeration}, \textit{instruction schedule} and
\textit{register allocation}, and the motivation for each is presented in the respective
subsections.


\subsection{Enumeration}

The name of this strategy comes from the fact that we are only concerned that the solutions
are different, not how they differ. During search the Gecode search engine will never
explore the same combination twice, and thus never generate two equal solutions. The
strategy is thus to not post any constraints at all and let the solver generate all
possible combinations. We \textit{enumerate} the solutions.

Unison can differ the solutions in four main ways:

\begin{itemize}
	\item The order of the operations is different
	\item Operands are connected to different registers
	\item Execute a copy using a different instruction (or not at all)
	\item Split live-ranges and spill temporaries differently
\end{itemize}

The results of this strategy serve as a baseline for the program. It is literally what
happens if we do nothing.

\subsection{Registers}

The strategy to diversify the register allocation of the resulting binaries is an attractive
one due to causing no run-time overhead. Consequently, if it introduces significant diversity
it is an excellent candidate. As register allocation is one of Unison's primary purpose it
feels like a natural strategy to explore.

There are two variables and one set that are of concern when diversifying the register
allocation. Their description from the Unison documentation are as follows:

\vspace{0.2cm}

\noindent\makebox[\textwidth]{
	\begin{tabular}{c|lr}
		\textbf{Type} & \textbf{Name} & \textbf{Description} \\ \hline
		\textbf{Set} & $P$ & The set of all operands in the program \\ \hline
		\multirow{2}{*}{\textbf{Variable}}
			& $x_p \in \{0, 1\}$ & whether operand \textit{p} is connected (0 is false and 1 is true) \\
			& $ry_p \in \mathbb{N}_0$ & register to which operand \textit{p} is assigned \\
	\end{tabular}
}

\vspace{0.2cm}

In order to disallow the same register allocation we post the constraint:

\vspace{0.2cm}
\noindent\makebox[\textwidth]{
	$\neg (\bigwedge\limits_{p \in P} ( (x_p = prev.x_p) \land (ry_p = prev.ry_p) ))$
}
\vspace{0.2cm}

Very important to note is that the variables preceded by \textit{prev.} (i.e $prev.x_p$ and
$prev.ry_p$) represents an actual value. More precisely the value that is part of the previous
solution. For example $prev.xy_p$ represents the register to which operand $p$ is assigned
in the previous solution. The variables not preceded by \textit{prev.} are variables in
the constraint programming sense and their corresponding domains make up the remaining
combinations to explore.

In words; We disallow the exact same combination of connected (used) operands and
operand to register mapping.

\subsection{Instruction Schedule}

Given how Unison functions diversifying the instruction schedule is an exciting strategy.
As mentioned in Section \ref{sec:unison} Unison explores optional copies. In practice this
means that during pre-processing optional \textit{operations} are inserted so not only
does Unison decide on the order the instructions are executed, but in a limited capacity
Unison also inserts instructions (or deems instructions unnecessary). For the purposes
of breaking gadgets shifting instructions can help immensely as an adversary is reliant
on the exact addresses of the gadgets.

There are two variables and one set of interest for this strategy. In the Unison
documentation they are described as follow:

\vspace{0.2cm}

\noindent\makebox[\textwidth]{
	\begin{tabular}{c|lr}
		\textbf{Type} & \textbf{Name} & \textbf{Description} \\ \hline
		\textbf{Set} & $O$ & The set of all operations in the program \\ \hline
		\multirow{2}{*}{\textbf{Variable}}
			& $a_o \in \{0, 1\}$ & whether operation \textit{o} is active (0 is false and 1 is true) \\
			& $c_o \in \mathbb{N}_0$ & issue cycle of operation \textit{o} \\
	\end{tabular}
}

\vspace{0.2cm}

In pseudo mathematical notation we want to post the constraint:

\vspace{0.2cm}
\noindent\makebox[\textwidth]{
	$\neg ( (\bigwedge\limits_{o \in O} (a_o = prev.a_o)) \land (\bigwedge\limits_{m \in \{O | prev.a_m = 1\}} (c_m = prev.c_m)) )$
}
\vspace{0.2cm}

Just as for the registers strategy constraint, the variables preceded by \textit{prev.}
represents the actual value that is part of the previous solution, whereas the ones not
preceded by \textit{prev.} are the constraint programming variables used when searching for
future solutions.

The constraint described with words is that we disallow the exact same combination of
active operations and their corresponding issue cycle. We are only concerned about the
issue cycle of the previous solution's active operations, but we do take into account that
future solutions can have the same instructions issued at the same cycles if it has fewer
or more active operations.

As mentioned in Section \ref{sec:unison}, some operations can be implemented by multiple
instructions. For the purpose of breaking gadgets we do not want to allow functionally
equivalent instructions. While they would indeed make the sequence of bytes differ, the
functionality would be the same and the gadget would survive.


\section{Branching Strategy}
\label{sec:branch_strategy}

The disUnison models uses the same branching strategy as the original Unison model. When
the search engine reaches a state where branching is necessary the first decision is to
assign the \textit{cost} (see Section \ref{sec:cost}) variable to its lowest possible
value. If the \textit{cost} variable is already assigned, the branching is done as
follows, in the order listed:

\begin{enumerate}
	\item assign the active operations. (the $a_o$ variable)
	\item assign which instruction should implement each operation.
	\item assign which temporary is connected to each operand. (the $y_p$ variable)
	\item assign which cycle each operation is issued. (the $c_o$ variable)
	\item assign which register is assigned to which temporary. (the $r_t$ variable)
\end{enumerate}

\section{Sampling Rate}
\label{sec:sampling_rate}

When exploring combinations with a constraint solver similar solutions are found close to
each-other (in-time). A factor in diversification might be to exploit this property
alongside the diversification strategy. Certain strategies might be favorable when
considering execution time but not particularly good at breaking gadgets. However, if
e.g 100 solutions are discarded between every emitted executable perhaps more gadgets are
broken.

In the original Unison model the cost variable is used both for branching and during the
branch and bound process. As mentioned in Section \ref{sec:branch_strategy}, in the
disUnison model it is still used for branching. Lower cost combinations are explores
first. However, future solutions are not bound to have lower cost. Discarding solutions
can thus impact performance in a negative way in the sense that the later versions might
have a higher cost, resulting in a wider cost distribution across all program versions.

Sampling rates of 1, 10, 100 and 1000 will be evaluated for every strategy, where a
sampling rate of 100 means that every 100th solution is kept. Generating 1000 versions
at a sampling rate of 100 would mean that 100000 solutions are explored, 1000 are emitted
and 99000 are discarded.

The number of possible combinations is of course not limitless. 1000 version at a sampling
rate of 1000 means that there needs to be at least 1000000 possible solutions. Unison works
at the function level and for every function there might not be 1000000 possible versions.

Total number of possible combination would be an interesting metric to evaluate, unfortunately
it varies widely between functions and for some it might require days of search. Empirically,
most functions in the suite to be used do have 1000000 versions, so 1000 versions appears
to be a good number of versions to generate.

For those function where 1000 versions cannot be generated for the given strategy and
sampling rate the ones that have been generated will be re-used so that 1000 program
versions can still be generated. More information about these functions can be seen in
appendix \ref{appendix:function_names}.

\section{Target Architecture}
\label{sec:arch}

Unison does not currently support the x86 or x86-64 architecture. Only ARM, Hexagon and MIPS
are supported \cite{unison-src}. None of the supported architectures are generally considered
when testing automated software diversity, but for the purposes of evaluating the use of
a systematic approach the supported architectures offers a glimpse at the potential. For
the experiment the code will be compiled for the Hexagon architecture.

Hexagon functions very differently from x86 but for our purposes targeting Hexagon will
still hint at the gadget-breaking potential of the systematic approach. After all,
breaking gadgets is about shifting, adding, removing or otherwise modifying \textit{any}
instruction in the program, and since the diversification strategies (see Section
\ref{sec:disUnison}) are applied globally they are supposedly equally effective regardless
of the placement or structure of the instruction.

\chapter{Experimental Setup}

In order to evaluate both the code generator and the generated code for each strategy a
population of programs for each strategy is required. In this section the data set, the
evaluation metrics and the process for generating the program populations will be
presented.

The experiment will be carried out on a computer running a 4-core Intel(R) Core(TM)
i7-4500U CPU @ 1.80GHz and with 8 gigabytes of memory.

\section{Data Set}

The data set to be used is part of the Unison test suite for the Hexagon architecture. In
total 23 functions will be used, each of which is from a benchmark in the SPEC2006\footnote{https://www.spec.org/cpu2006/ (visited on 21/06/2018)}
suite. These function will together make up a \textit{program}. Since they do not make up
a complete executable they cannot be linked nor executed. Linking will instead be simulated
by placing them in the same order every time to ensure a fair comparison between
strategies and sampling rates.

As mentioned in Section \ref{sec:unison-model} Unison works on the function level, and so
does disUnison. 1000 versions of each function will be generated and labeled from 0 through
999. Version 0 of each function will make up program version 0, version 1 of each function
will make up program version 1 and so forth. That is, one program version consists of 23
functions, and there will be a total of 1000 program versions for each strategy and
sampling rate, yielding a total of 12 programs with 1000 versions each.

\section{Metrics}
\label{sec:metrics}

The metrics to be evaluated are surviving gadgets, \textit{cost} (See \ref{sec:cost}),
both speed and size, and the execution time of the code generator.

The \textit{cost} metric is calculated by a tool called \textit{uni analyze}, which is
part of the Unison toolchain. It accepts the LLVM MIR of a function as input and outputs
the estimated cost for each optimization goal as described in Section \ref{sec:cost}. The
cost in terms of size of a program version will be calculated as the sum of the cost for
each function version that makes up the program version in question (i.e. the total number
of instructions in the program version). The cost in speed of a program version will be
calculated as the  mean between the cost of the corresponding functions.

For the experiment the optimization goal will be speed, and thus solutions that are
estimated to execute faster are generated first. However, both cost in speed and cost in
size will be presented in the results section.

Surviving gadgets will be calculated as the ratio of which each gadget appears among the
population of program versions. All gadgets present in any of the 1000 program versions
will be enumerated. The ratio is calculated as the number of occurrences of every unique
gadget divided by the number of versions, which in this case is 1000. In other words a
ratio of 100\% means that the gadget appears in all version and the strategy was not
effective at breaking that gadget. Similarly a very low ratio (close to 0\%) would mean
that the strategy was effective as the gadget only appears in a small number of the
programs.

\section{Generation Process}

As mentioned in Section \ref{sec:sampling_rate} most functions in the suite has 1000000
different versions so all strategies will be tested for sampling rates up to 1000. There
is a lot of disk I/O involved in generating all versions and even more so when calculating
the metrics. Thus, sampling a population of 1000 is empirically a good choice to keep the
calculations relatively fast.

Our process to generate our test data is as follows:

\begin{enumerate}
	\item For every function generate 1000 versions with "speed" as the optimization goal.
	\item Version 0 of every function will make up program 0, version 1 will make up program
		1 and so forth.
	\item Calculate/record our metrics.
		\begin{itemize}
			\item Execution time of code generator
			\item Cost
				\begin{itemize}
					\item speed
					\item size
				\end{itemize}
			\item Frequency of surviving gadgets
		\end{itemize}
	\item Repeat for all strategies.
		\begin{itemize}
			\item Enumerate.
			\item Registers.
			\item Schedule.
		\end{itemize}
	\item Repeat for all sampling rates.
		\begin{itemize}
			\item 1.
			\item 10.
			\item 100.
			\item 1000.
		\end{itemize}
\end{enumerate}

In other words, for every strategy and sampling 1000 versions of each function will be
generated. Each combination of strategy and sampling rate will thus have 23000 function
versions assosiated with it. Version 0 of every function for each strategy and sampling
rate will make up program version 0 for the corresponding strategy and sampling rate
combination. Version 1 will make up program version 1 and so forth. Each program version
will consist of 23 functions so the process will yield 1000 program versions for each
combination of strategy and sampling rate.

The resulting test data will be 12 programs, one for each combination of strategy and
sampling rate. Each of these 12 programs represents a population of 1000 different versions,
for a total of 12000 program versions. Each version consists of 23 functions for a total
of 276000 function versions. The result will emphasize comparison between the sampling
rates for each strategy as well as the difference between the strategies.


% Result
\chapter{Results}

\section{Surviving Gadgets}

\begin{figure}[htp]
	\centering
	\includegraphics[width=\textwidth,height=\textheight]{results/figures/gadgets}
	\caption{The ratio of occurence for each gadget broken down by strategy and sampling rate.
	Each bar shows the occurence ratio for one particular gadget.}
	\label{fig:gadgets}
\end{figure}

Figure \ref{fig:gadgets} shows the occurence ratio of each gadget for the different
strategies and sampling rates. There were around 21000 gadgets in total for every program
across all versions. The x-axis shows a gadget id and there is no definitive correlation
between the gadgets across strategies and sampling rates. I.e gadget 0 for the enumerate
strategy at sampling rate 1 is not necessarily the same gadget 0 as the registers strategy
at sampling rate 10.

As seen in figure \ref{fig:gadgets}, neither the enumeration nor the registers strategy
were particularly effective at breaking gadgets for any sampling rate. There is a slight
improvement for higher sampling rates but it is not particularly impressive even at
sampling rate 1000. There are still many gadgets that appear in all programs.

Curiously, there seems to be little difference between sampling rate 10 and 100 for both
enumerate and registers. Sampling rate of 100 appears to be even worse at breaking gadgets
than sampling rate 10.

The schedule strategy seems to have performed very well. Not a single gadget was present
in even 50\% of all versions for any sampling rate. The strategy is even more effective at
higher sampling rates, which is in accordance with the expected behaviour described in
section \ref{sec:sampling_rate}.

\section{Cost}

The estimated cost of each program is shown in Figure \ref{fig:cost}. The dotted red line
shows the cost of the LLVM solution when calculated in the same manner.

\begin{figure}[h]
	\centering
	\includegraphics[width=\textwidth,height=0.5\textheight]{results/figures/cost}
	\caption{The cost distributions for every strategy and sampling rate. The cost of the LLVM solution is included for reference.}
	\label{fig:cost}
\end{figure}

All strategies perform better for lower sampling rates. As described in section
\ref{sec:performance} this is expected. Enumerate and registers perform equally well and
for sampling sizes of 1 and 10 they have a lower cost than the LLVM solution. The schedule
strategy seems to incur a significant overhead compared to the LLVM solution for all
sampling rates.

Interesting to note is that all strategies and sampling rates have found a solution with
an equally low cost. This is presumably the very first solution found; When no strategy
related constraints have been posted yet.


\section{Discussion}

What is interesting is that the registers strategy actually seem to perform \textit{worse}
than the enumerate strategy for the 10 and 100 sampling rates. This is surprising
because the versions of the registers strategy is theoretically a subset of versions of
the enumerate strategy.

% Remember to mention the hypothesis!!!

\section{Conclusion}

% Calculate the performance impact for carefully. Perhaps percentages in it's own plot?
To summarize the results and discussion it seems fair to draw the conclusion that enumerate
and registers, while incuring low overhead breaks very few gadgets and are thus not very
useful. The schedule strategy, however, performs very well when breaking gadgets but less
so regarding the cost. The cost of the schedule strategy varies greatly between sampling
rates, and for the lowest rate there is about 5\% performance impact for the mean.


% Discussion, Conclusion and Future Work
\chapter{Discussion, Conclusion and Future Work}
\label{chapter:discussion}

% Implementation
The core of the disUnison model, the three strategies, are implemented in just a handful
lines of code and, thanks to being an extension of the Unison model, there is never a risk
of breaking the functionality of the generated code. This is not to say that every
diversification strategy allows for proper executable code to be generated, but thanks to
the nature of constraint solving the result will either be all possible solutions
(given enough execution time) or a proof that no solution, and thus no proper executable,
exists.

Regardless of what the selection of strategies may indicate the possibilities for
diversification are far broader than when approaching the problem in terms of register
allocation and instruction scheduling as separate procedures. It is important to keep in
mind that more unorthodox strategies that exploit the combined approach might be even
more performant. As mentioned in \ref{sec:unison-model} the Unison model consists of more
variables than the 4 explored in this experiment, all of which offers potential for
diversity.

% Cost
By exploring lower cost solutions first and applying tiny, incremental changes between
solutions the performance of the resulting code is widely distributed but also relatively
good. Only the highest sampling rates for the schedule strategy incurs a significant
overhead. There is also an opportunity to add constraints to the model that steers this
cost in some direction. There is opportunity to limit our executables to have a smaller cost
than the LLVM solution, or to only incur an overhead of e.g 5\% compared to the LLVM
solution if we want more versions. By not randomizing we have full control of the process
and can limit the resulting code in whatever way is appropriate.

Unison (and disUnison) accepts the basic LLVM-solution as an optional parameter, and can
post constraints to only generate solutions that are \textit{better}. In other words, we
can generate executables with zero overhead \textit{with respect to LLVM's solution}.
Certain strategies would of course have an overhead, but with respect to the
\textit{optimal solution}. As this would limit the number of possible version this
optional parameter was not used during the experiment. It is however an exciting factor to
consider for future work.

There is a constant trade-off between diversity and execution overhead when generating
diverse populations of executables and the systematic approach is not excluded from this.
From the results we can deduce that there is a correlation that a more diverse population
in terms of gadgets incurs a wider distribution and a higher mean for the cost metric even
for the systematic approach. However, when comparing sampling rates 10 and 100 of the
enumerate and registers strategy the higher sampling rate generates a noticeably more
diverse population, but the mean cost in speed is virtually unchanged and the code size is
only increased by a handful of percentage points. Perhaps a more advanced strategy can
lessen this gap even more and still provide a diverse population of executables.

% Thread to validity
\textcite{large-scale-automated} found 433.milc from the SPEC2006 benchmark suite to be
representative in terms of surviving gadgets. Unfortunately 433.milc was too large for the
experiment and the Unison test suite had to be used instead. Most the functions were too
large to find even one solution (one searching for about one hour). This is an obvious
problem for any practical purpose of the systematic approach. One solution to this problem
is to modify the search heuristics of the disUnison model even further. Unfortunately it
would be difficult, if not impossible, to find an optimal, generally-applicable search
strategy.

For a more proper comparison of the systematic approach tests would have to be repeated for
a more comprehensive data set and target the x86-64 architecture. As mentioned in Section
\ref{sec:arch} Unison in its current state does not support the x86-64 architecture. If or
when Unison or a similar tool implements support for the x86-64 architecture, the
experiment would have be repeated on to test whether or not the strategies are equally
performant.

% Conclusion
Using a constraint solver to generate diverse binaries is an attractive approach given
the ease of implementation and the quality of the generated code. The resulting population
of diversified programs shows that the systematic approach has great potential at breaking
gadgets as well as providing great control of the incurred overhead. However, the
shortcomings of the experiment and the tool are a testament to the future work required.


% Bibliography
\printbibliography[heading=bibintoc] % Print the bibliography (and make it appear in the table of contents)

% Appendices
\appendix

% Appendix A
\chapter{Functions}
\label{appendix:function_names}
% For all functions not present here 1000 versions was found. 

\begin{Table}
	\begin{tabular}{c|c|c}
		\textbf{Function Name} & & \textbf{Strategy} \textbf{# versions(1,10,100,1000)} \ \hline
		gcc.insn-output.output\_51 & sched & 743 \
		gcc.jump.unsigned\_condition & sched & 243, 23, 3, 1 \
		gobmk.barriers.autohelperbarrierspat145 & sched & 1000, 1000, 485, 53 \
		gobmk.interface.init\_gnugo & difference & 792, 80, 8, 1 \
		gobmk.interface.init\_gnugo & registers & 1, 1, 1, 1 \
		gobmk.interface.init\_gnugo & schedule & 792, 80, 8, 1 \
		gobmk.owl\_attackpat.autohelperowl\_attackpat68 & schedule & 1000, 1000, 1000, 917 \
		gobmk.patterns.autohelperpat1088 & schedule & (100, 1000, 1000, 842) \
		gobmk.patterns.autohelperpat301 & schedule & (1000, 1000, 1000, 936) \
		hmmer.tophits.AllocFancyAli & all & (0,0,0,0) \
		gcc.xexit.xexit & all & (0,0,0,0) \
		sphinx3.profile.ptmr\_init & registers & (17, 17, 17, 17) \
	\end{tabular}
\end{Table}


% Appendix B
\chapter{ROP Exploit}
\label{appendix:rop-exploit}

Python code that executes a ROP exploit for the ROP emporium 64-bit split challenge.\footnote{https://ropemporium.com/challenge/split.html}.
Requires the pwntools package\footnote{https://github.com/Gallopsled/pwntools} and that
the challenge files are located in the same directory as the script.

\lstinputlisting[tabsize=2,frame=single,breaklines=true,showstringspaces=false]
{appendices/rop-exploit.py}

\end{document}
